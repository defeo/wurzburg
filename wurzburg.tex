\documentclass[10pt]{article}

\usepackage[a4paper]{geometry}
\usepackage[english]{babel}
\usepackage{array}
\usepackage{amsmath,amsthm,amsfonts,amssymb}
\usepackage{unicode}
\usepackage{subcaption}
\usepackage[type={CC},modifier={by-nc},imagemodifier={-eu},version={4.0},imagewidth=5em]{doclicense}
\usepackage{fancyhdr}

\usepackage{algorithmic}
\renewcommand{\algorithmicrequire}{\textbf{Input:}}
\renewcommand{\algorithmicensure}{\textbf{Output:}}
\algsetup{linenodelimiter=.}

\usepackage[pdfusetitle]{hyperref}
\hypersetup{
  unicode=true,
  colorlinks=true,
  citecolor=blue!70!black,
  filecolor=black,
  linkcolor=red!70!black,
  urlcolor=blue,
  pdfstartview={FitH},
  pdfauthor={Luca De Feo},
  pdfsubject={Mathematics},
  pdfkeywords={Cryptography, Number theory, Elliptic curves, Isogenies},
}

\usepackage{tikz}
\usetikzlibrary{arrows,matrix,decorations,decorations.text,decorations.pathmorphing,calc}
\pgfkeys{/triangle/.code=\tikzset{x={(-0.5cm,-0.866cm)},y={(1cm,0cm)}}}
\pgfkeys{/lattice/.code n args={4}{\tikzset{cm={#1,#2,#3,#4,(0,0)}}}}

\newcommand{\axes}[4]{
  \clip (#1,#3) rectangle (#2,#4);
  \draw [thin, gray, -latex] (#1,0) -- (#2,0);% Draw x axis
  \draw [thin, gray, -latex] (0,#3) -- (0,#4);% Draw y axis
}

\newcommand{\lattice}[3][2pt]{
  \draw[style=help lines,dashed] (#2-1,#2-1) grid[step=1] (#3+1,#3+1);
  \foreach \x in {#2,...,#3}{
    \foreach \y in {#2,...,#3}{
      \node[draw,circle,inner sep=#1,fill] at (\x,\y) {};
      % Places a dot at those points
    }
  }
}

% theorem environments
\theoremstyle{plain}
\newtheorem{theorem}{Theorem}
\newtheorem{lemma}[theorem]{Lemma}
\newtheorem{corollary}[theorem]{Corollary}
\newtheorem{proposition}[theorem]{Proposition}
\theoremstyle{definition}
\newtheorem{definition}[theorem]{Definition}
\newtheorem{example}[theorem]{Example}
\newtheorem{problem}{Problem}
\newtheorem{exercise}{Exercise}[part]

\DeclareMathOperator{\Aut}{Aut}
\DeclareMathOperator{\End}{End} % endomorphism ring
\DeclareMathOperator{\Tr}{Tr} % finite field trace
\DeclareMathOperator{\Gal}{Gal} % Galois group
\DeclareMathOperator{\ord}{ord} % order of an element
\DeclareMathOperator{\lcm}{lcm} % least common multiple
\DeclareMathOperator{\norm}{N} % norm
\DeclareMathOperator{\loglog}{loglog}
\DeclareMathOperator{\im}{Im}
\DeclareMathOperator{\GL}{GL}
\DeclareMathOperator{\SL}{SL}
\DeclareMathOperator{\Cl}{Cl}
\DeclareMathOperator{\Ell}{Ell}

\def\A{\ensuremath{\mathbb{A}}}
\def\P{\ensuremath{\mathbb{P}}}
\def\F{\ensuremath{\mathbb{F}}}
\def\O{\ensuremath{\mathcal{O}}}
\def\tildO{\ensuremath{\tilde{O}}}
\def\euler{\ensuremath{\varphi}}
\def\a{\ensuremath{\mathfrak{a}}}
\def\mat#1{\left(\begin{smallmatrix}#1\end{smallmatrix}\right)}
\newcommand{\leg}[2]{\left(\frac{#1}{#2}\right)}

\newcommand{\bl}[1]{\textcolor{blue}{#1}}
\newcommand{\rd}[1]{\textcolor{red}{#1}}

\title{Isogeny Graphs in Cryptography}
\author{Luca De Feo\\
  Universit\'e Paris Saclay -- UVSQ\\
  \url{https://defeo.lu/}}
\date{Graph Theory Meets Cryptography\\
  July 29 -- August 2, 2019, Wurzb\"urg, Germany}

\begin{document}
\maketitle
\thispagestyle{fancy}
\renewcommand{\headrulewidth}{0pt}
\renewcommand{\footrulewidth}{0.4pt}
\cfoot{\doclicenseThis}
\lfoot{\LaTeX{} source code available at \url{https://github.com/defeo/wurzburg/}.}

\section*{Introduction}

These lectures notes were written for the summer school 
\emph{Graph Theory Meets Cryptography} in Wurzb\"urg, Germany. %

The presentation is divided in four parts, roughly corresponding to
the four lectures given. %

{
  \hypersetup{linkcolor=black}
  \setcounter{tocdepth}{1}
  \tableofcontents
}

%%%%%%%%%%%%%%%%%%%%%%%%%%%%%%%%%%%%%%%%%%%%%%%%%%%%%%

\clearpage
\part{Elliptic curves and isogenies}

In this part, we review the basic and not-so-basic theory of elliptic
curves. %
Our goal is to summarize the fundamental theorems necessary to
understanding the foundations of isogeny based cryptography. %
A proper treatment of the material covered here would require more
than one book, we thus skip proofs and lots of details to go straight
to the useful theorems. %
The reader in search of a more comprehensive treatment will find more
details~\cite{silverman:elliptic,silverman:advanced,lang1987elliptic,neukirch2013algebraic}. %

Throughout this part we let $k$ be a field, and we denote by $\bar{k}$
its algebraic closure. %

\section{Elliptic curves}

Elliptic curves are projective curves of genus 1 with a distinguished
point. %
Projective space initially appeared through the process of adding
\emph{points at infinity}, as a method to understand the geometry of
projections (also known as \emph{perspective} in classical
painting). %
In modern terms, we define projective space as the collection of all
lines in affine space passing through the origin.

\begin{definition}[Projective space]
  The \emph{projective space of dimension $n$}, denoted by $\P^n$ or
  $\P^n(\bar{k})$, is the set of all $(n+1)$-tuples
  \[(x_0,\dots,x_n) ∈ \bar{k}^{n+1}\] %
  such that $(x_0,\dots,x_n) ≠ (0,\dots,0)$, taken modulo the
  equivalence relation
  \[(x_0,\dots,x_n) \sim (y_0,\dots,y_n)\] %
  if and only if there exists $λ\in\bar{k}$ such that
  $x_i=λ_iy_i$ for all $i$.
\end{definition}

The equivalence class of a projective point $(x_0,\dots,x_n)$ is
customarily denoted by $(x_0:\cdots:x_n)$. %
The set of the \emph{$k$-rational points}, denoted by $\P^n(k)$, is
defined as
\[\P^n(k) = \left\{(x_0:\cdots:x_n)∈\P^n\;\middle|\; x_i ∈ k \text{ for all $i$}\right\}.\] %
By fixing arbitrarily the coordinate $x_n=0$, we define a projective
space of dimension $n-1$, which we call the \emph{hyperplane at
  infinity}; its points are called \emph{points at infinity}.

From now on we suppose that the field $k$ has characteristic different
from $2$ and $3$. %
This has the merit of greatly simplifying the representation of an
elliptic curve. %
For a general definition, see~\cite[Chap.~III]{silverman:elliptic}.

\begin{definition}[Weierstrass equation]
  An \emph{elliptic curve} defined over $k$ is the locus in
  $\P^2(\bar{k})$ of an equation
  \begin{equation}
    \label{eq:weierstrass}
    Y^2Z = X^3 + aXZ^2 + bZ^3,    
  \end{equation}
  with $a,b∈k$ and $4a^3+27b^2\ne0$.

  The point $(0:1:0)$ is the only point on the line $Z=0$; it is
  called the \emph{point at infinity} of the curve.
\end{definition}

It is customary to write Eq.~\eqref{eq:weierstrass} in \emph{affine
  form}. %
By defining the coordinates $x=X/Z$ and $y=Y/Z$, we equivalently
define the elliptic curve as the locus of the equation
\[y^2 = x^3 + ax +b,\]
plus the point at infinity $\O=(0:1:0)$.

In characteristic different from $2$ and $3$, we can show that any
projective curve of genus $1$ with a distinguished point $\O$ is
isomorphic to a Weierstrass equation by sending $\O$ onto the point at
infinity $(0:1:0)$.

\begin{figure}
  \centering
  \hfill
  %% 
  \begin{tikzpicture}[domain=-2.4566:4,samples=100,yscale=3/8,xscale=3/4]
    \draw plot (\x,{sqrt(\x*\x*\x-4*\x+5)});
    \draw plot (\x,{-sqrt(\x*\x*\x-4*\x+5)});

    \draw[thin,gray,-latex] (0,-7) -- (0,7);
    \draw[thin,gray,-latex] (-3,0) -- (4,0);

    \draw (-3,1) -- (4,8/3+3);
    \begin{scope}[every node/.style={draw,circle,inner sep=1pt,fill},cm={1,2/3,0,0,(0,3)}]
      \node at (-2.287980,0) {};
      \node at (-0.535051,0) {};
      \node at (3.267475,0) {};
    \end{scope}
    \begin{scope}[every node/.style={yshift=0.3cm},cm={1,2/3,0,0,(0,3)}]
      \node at (-2.287980,0) {$P$};
      \node at (-0.535051,0) {$Q$};
      \node at (3.267475,0) {$R$};
    \end{scope}

    \draw[dashed] (3.267475,3.267475*2/3+3) -- (3.267475,-3.267475*2/3-3) 
    node[draw,circle,inner sep=1pt,fill] {}
    node[xshift=-0.1cm,anchor=east] {$P+Q$};
  \end{tikzpicture}
  %% 
  \hfill
  %% 
  \begin{tikzpicture}[domain=-2.4566:4,samples=100,yscale=3/8,xscale=3/4]
    \draw plot (\x,{sqrt(\x*\x*\x-4*\x+5)});
    \draw plot (\x,{-sqrt(\x*\x*\x-4*\x+5)});

    \draw[thin,gray,-latex] (0,-7) -- (0,7);
    \draw[thin,gray,-latex] (-3,0) -- (4,0);
    
    \def\c{3.269524}
    \def\P{-1.398674}
    \def\R{2.908459}
    \draw (-3,-1+\c) -- (4,4/3+\c);
    \begin{scope}[every node/.style={draw,circle,inner sep=1pt,fill},cm={1,1/3,0,0,(0,3.269524)}]
      \node at (\P,0) {};
      \node at (\R,0) {};
    \end{scope}
    \begin{scope}[every node/.style={yshift=0.3cm},cm={1,1/3,0,0,(0,3.269524)}]
      \node at (\P,0) {$P$};
      \node at (\R,0) {$R$};
    \end{scope}

    \draw[dashed] (\R,\R/3+\c) -- (\R,-\R/3-\c) 
    node[draw,circle,inner sep=1pt,fill] {}
    node[xshift=-0.1cm,anchor=east] {$[2]P$};
  \end{tikzpicture}
  %%
  \hfill
  \strut
  
  \caption{An elliptic curve defined over $ℝ$, and the geometric
    representation of its group law.}
  \label{fig:weierstrass}
\end{figure}

Now, since any elliptic curve is defined by a cubic equation, Bezout's
theorem tells us that any line in $\P^2$ intersects the curve in
exactly three points, taken with multiplicity. %
We define a group law by requiring that three co-linear points sum to
zero. %

\begin{definition}
  Let $E\;:\;y^2=x^3+ax+b$ be an elliptic curve. Let $P_1=(x_1,y_1)$
  and $P_2=(x_2,y_2)$ be two points on $E$ different from the point at
  infinity, then we define a composition law $⊕$ on $E$ as
  follows:
  \begin{itemize}
  \item $P ⊕ \O = \O ⊕ P = P$ for any point $P∈E$;
  \item If $x_1=x_2$ and $y_1=-y_2$, then $P_1⊕P_2 = \O$;
  \item Otherwise set
    \[λ =
      \begin{cases}
        \frac{y_2-y_1}{x_2-x_1} &\text{if $P≠Q$,}\\
        \frac{3x_1^2+a}{2y_1} &\text{if $P=Q$,}
      \end{cases}
    \]
    then the point $(P_1⊕P_2)=(x_3,y_3)$ is defined by
    \begin{align*}
      x_3 &= λ^2 - x_1 - x_2,\\
      y_3 &= -λx_3 - y_1 + λx_1.
    \end{align*}
  \end{itemize}
\end{definition}

It can be shown that the above law defines an Abelian group, thus we
will simply write $+$ for $⊕$. %
The $n$-th scalar multiple of a point $P$ will be denoted by $[n]P$. %
When $E$ is defined over $k$, the subgroup of its \emph{rational
  points over $k$} is customarily denoted $E(k)$. %
Figure~\ref{fig:weierstrass} shows a graphical depiction of the group
law on an elliptic curve defined over $ℝ$.

We now turn to the group structure of elliptic curves. %
The torsion part is easily characterized.

\begin{proposition}
  Let $E$ be an elliptic curve defined over an algebraically closed
  field $k$, and let $m≠0$ be an integer. %
  The $m$-torsion group of $E$, denoted by $E[m]$, has the following
  structure:
  \begin{itemize}
  \item $E[m] ≃ (ℤ/mℤ)^2$ if the characteristic of $k$ does not divide
    $m$;
  \item If $p>0$ is the characteristic of $k$, then 
    \[E[p^i] ≃
      \begin{cases}
        ℤ/p^iℤ & \text{for any $i≥0$, or}\\
        \{\O\} & \text{for any $i≥0$.}
      \end{cases}
    \]
  \end{itemize}
\end{proposition}
\begin{proof}
  See~\cite[Coro.~6.4]{silverman:elliptic}. For the characteristic $0$
  case see also Section~\ref{sec:elliptic-curves-over}.
\end{proof}

For curves defined over a field of positive characteristic $p$, the
case $E[p]≃ℤ/pℤ$ is called \emph{ordinary}, while the case
$E[p]≃\{\O\}$ is called \emph{supersingular}. %
We shall see an alternative characterization of supersingularity in
the next section.

The free part of the group is much harder to characterize. %
We have some partial results for elliptic curves over number fields.

\begin{theorem}[Mordell-Weil]
  Let $k$ be a number field, the group $E(k)$ is finitely generated.
\end{theorem}

However the exact determination of the rank of $E(k)$ is somewhat
elusive: we have algorithms to compute the rank of most elliptic
curves over number fields; however, an exact formula for such rank is
the object of the
\href{https://en.wikipedia.org/wiki/Birch_and_Swinnerton-Dyer_conjecture}{\it
  Birch and Swinnerton-Dyer conjecture}, one of the
\href{https://en.wikipedia.org/wiki/Millennium_Prize_Problems}{\it
  Clay Millenium Prize Problems}.

\section{Maps between elliptic curves}

Finally, we focus on maps between elliptic curves. %
We are mostly interested in maps that preserve both facets of elliptic
curves: as projective varieties, and as groups. %

We first look into invertible algebraic maps, that is linear changes
of coordinates that preserve the Weierstrass form of the equation. %
Because linear maps preserve lines, it is immediate that they also
preserve the group law. %
It is easily verified that the only such maps take the form
\[(x,y) \mapsto (u^2x', u^3y')\] %
for some $u∈\bar{k}$, thus defining an \emph{isomorphism} between the
curve $y^2=x^3+au^4x+bu^6$ and the curve $(y')^2 = (x')^3 + ax' +
b$. %
Isomorphism classes are traditionally encoded by an invariant, whose
origins can be traced back to complex analysis.

\begin{proposition}[$j$-invariant]
  \label{th:j}
  Let $E:y^2=x^3+ax+b$ be an elliptic curve, and define the
  \emph{$j$-invariant} of $E$ as
  \[j(E) = 1728\frac{4a^3}{4a^3+27b^2}.\] %
  Two curves are isomorphic over the algebraic closure $\bar{k}$ if
  and only if they have the same $j$-invariant.
\end{proposition}

Note that if two curves defined over $k$ are isomorphic over
$\bar{k}$, they are so over an extension of $k$ of degree dividing
$6$. %
An isomorphism between two elliptic curves defined over $k$, that is
itself not defined over $k$ is called a \emph{twist}. %
Any curve has a \emph{quadratic twist}, unique up to isomorphism,
obtained by taking $u∉k$ such that $u^2∈k$. %
The two curves of $j$-invariant $0$ and $1728$ also have \emph{cubic},
\emph{sextic} and \emph{quartic twists}.

A surjective group morphism, not necessarily invertible, between two
elliptic curves is called an \emph{isogeny}. %
It turns out that isogenies are algebraic maps as well.

\begin{theorem}
  Let $E,E'$ be two elliptic curves, and let $\phi:E→E$ be a map between
  them. %
  The following conditions are equivalent:
  \begin{enumerate}
  \item $\phi$ is a surjective group morphism,
  \item $\phi$ is a group morphism with finite kernel,
  \item $\phi$ is a non-constant algebraic map of projective varieties
    sending the point at infinity of $E$ onto the point at infinity of
    $E'$.
  \end{enumerate}
\end{theorem}
\begin{proof}
  See~\cite[III, Th.~4.8]{silverman:elliptic}.
\end{proof}

Two curves are called \emph{isogenous} if there exists an isogeny
between them. %
We shall see later that this is an equivalence relation.

Isogenies from a curve to itself are called \emph{endomorphisms}. %
The prototypical endomorphism is the multiplication-by-$m$
endomorphism defined by
\[[m]\;:\; P \mapsto [m]P.\] %
Its kernel is exactly the $m$-th torsion subgroup $E[m]$. %

Since they are algebraic group morphisms, we can define addition of
isogenies by $(ϕ+ψ)(P) = ϕ(P)+ψ(P)$, and the resulting map is still an
isogeny. %
Thus, by including the constant map that sends every point to the
point at infinity, the set of isogenies $E\to E'$ forms a group. %
Additionally, endomorphisms $E\to E$ support composition, distributing
over addition, hence the set of all endomorphisms forms a ring,
denoted by $\End(E)$.%
\footnote{In short, isogenies are the morphisms in the Abelian
  category of elliptic curves.}

Since each $m∈ℤ$ defines a different multiplication-by-$m$
endomorphism, clearly $ℤ⊂\End(E)$. %
But can $\End(E)$ be larger? %
We shall now give a complete characterization of the endomorphism ring
for any elliptic curve.

\begin{definition}[Order]
  \label{def:order}
  Let $K$ be a finitely generated $ℚ$-algebra. %
  An \emph{order} $\O⊂K$ is a subring of $K$ that is a finitely
  generated $ℤ$-module, and that contains a $ℚ$-basis for $K$.
\end{definition}

The prototypical example of order is the ring of integers $\O_K$ of a
number field $K$. %
It turns out that $\O_K$ is the \emph{maximal order} of $K$, i.e., it
contains any other order of $K$. %
We shall discuss this case in depth in Section~\ref{sec:compl-mult}.

\begin{definition}[Quaternion algebra]
  A \emph{quaternion algebra} is an algebra of the form
  \[K = ℚ + αℚ + βℚ + αβℚ,\]
  where the generators satisfy the relations
  \[α^2,β^2∈ℚ, \quad α^2<0, \quad β^2 < 0, \quad βα=-αβ.\]
\end{definition}

\begin{theorem}[Deuring]
  Let $E$ be an elliptic curve defined over a field $k$ of
  characteristic $p$. %
  The ring $\End(E)$ is isomorphic to one of the following:
  \begin{itemize}
  \item $ℤ$, only if $p=0$;
  \item An order $\O$ in a quadratic imaginary field (a number field
    of the form $ℚ(\sqrt{-D})$ for some $D>0$); in this case we say
    that $E$ has \emph{complex multiplication} by $\O$;
  \item Only if $p>0$, a maximal order in a quaternion algebra
    ramified at $p$ and $∞$; in this case we say that $E$ is
    \emph{supersingular}.
  \end{itemize}
\end{theorem}
\begin{proof}
  See~\cite[III, Coro.~9.4]{silverman:elliptic}
  and~\cite{kohel}.
\end{proof}

In positive characteristic, a curve that is not supersingular is
called \emph{ordinary}; we shall see that it necessarily has complex
multiplication. %



\section{Elliptic curves over $ℂ$}
\label{sec:elliptic-curves-over}

To better understand elliptic curves and their morphisms, we take a
moment now to specialize them to the complex numbers.

\begin{definition}[Complex lattice]
  A \emph{complex lattice} $Λ$ is a discrete subgroup of $ℂ$ that
  contains an $ℝ$-basis of $ℂ$.
\end{definition}

Explicitly, a complex lattice is generated by a \emph{basis}
$(ω_1,ω_2)$, such that $ω_1≠λω_2$ for any $λ∈ℝ$, as
\[Λ = ω_1ℤ + ω_2ℤ.\] %
Up to exchanging $ω_1$ and $ω_2$, we can assume that $\im(ω_1/ω_2)>0$;
we then say that the basis has \emph{positive orientation}. %
A positively oriented basis is obviously not unique, though.

\begin{proposition}
  \label{th:basis-change}
  Let $Λ$ be a complex lattice, and let $(ω_1,ω_2)$ be a positively
  oriented basis, then any other positively oriented basis
  $(ω_1',ω_2')$ is of the form
  \begin{align*}
    ω_1' &= aω_1 + bω_2,\\
    ω_1' &= cω_1 + dω_2,
  \end{align*}
  for some matrix
  $\left(\begin{smallmatrix}a&b\\c&d\end{smallmatrix}\right)∈\SL_2(ℤ)$.
\end{proposition}
\begin{proof}
  See~\cite[I, Lem.~2.4]{silverman:advanced}.
\end{proof}

\begin{definition}[Complex torus]
  Let $Λ$ be a complex lattice, the quotient $ℂ/Λ$ is called a
  \emph{complex torus}.
\end{definition}

\begin{figure}
  \centering
  \begin{tikzpicture}[scale=2]
    \axes{-1}{3.5}{-0.5}{3}

    \begin{scope}[/lattice={1}{0.2}{0.4}{0.7}]
      \draw[fill,black!10] (0,0) -- (1,0) -- (1,1) -- (0,1) -- (0,0);
      \node at (0.5,0.5) {$ℂ/Λ$};
      \node at (0.9,-0.1) {$ω_2$};
      \node at (-0.1,0.9) {$ω_1$};

      \lattice{-3}{4}
    \end{scope}  
  \end{tikzpicture}

  \caption{A complex lattice (black dots) and its associated complex
    torus (grayed \emph{fundamental domain}).}
  \label{fig:lattice}
\end{figure}

A convex set of class representatives of $ℂ/Λ$ is called a
\emph{fundamental parallelogram}. %
Figure~\ref{fig:lattice} shows a complex lattice generated by a
(positively oriented) basis $(ω_1,ω_2)$, together with a fundamental
parallelogram for $ℂ/(ω_1,ω_2)$. %
The additive group structure of $ℂ$ carries over to $ℂ/Λ$, and can be
graphically represented as operations on points inside a fundamental
parallelogram. %
This is illustrated in Figure~\ref{fig:lattice-arith}.

\begin{figure}
  \centering
  \begin{tikzpicture}[scale=1.8]
    \axes{-0.5}{3.5}{-0.5}{3}

    \begin{scope}[/lattice={1}{0.2}{0.4}{0.7}]
      \lattice{-3}{4}

      \node[red] at (0.7,0.65) {$a$}; 
      \node[draw,circle,inner sep=1pt,fill,red] at (0.8,0.5) {};
      \node[red] at (0.2,0.9) {$b$}; 
      \node[draw,circle,inner sep=1pt,fill,red] at (0.3,0.7) {};
      
      \node[draw,circle,inner sep=1pt,fill,red] at (1.1,1.2) {};

      \draw[red,thin,dotted] (0,0) -- (0.8,0.5) -- (1.1,1.2)
      (0,0) -- (0.3,0.7) -- (1.1,1.2);          

      \node[red] at (0.2,0.3) {$a+b$}; 
      \node[draw,circle,inner sep=1pt,fill,red] at (0.1,0.2) {};
    \end{scope}  
  \end{tikzpicture}
  %%
  \hfill
  %%
  \begin{tikzpicture}[scale=1.8]
    \axes{-0.5}{3.5}{-0.5}{3}

    \begin{scope}[/lattice={1}{0.2}{0.4}{0.7}]
      \lattice{-3}{4}
      
      \node[red,yshift=0.2cm] at (0.8,0.6) {$a$}; 
      \draw[red] (0.8,0.6) node[fill,circle,inner sep=1pt] {};

      \draw[red,dotted] (0,0) -- (1.6,1.2) node[fill,circle,inner sep=1pt] {} 
      -- (2.4,1.8) node[fill,circle,inner sep=1pt] {};

      \node[red,yshift=0.3cm] at (0.4,0.8) {$[3]a$}; 
      \draw[red] (0.4,0.8) node[fill,circle,inner sep=1pt] {};
    \end{scope}
  \end{tikzpicture}
  \caption{Addition (left) and scalar multiplication (right) of points
    in a complex torus $ℂ/Λ$.}
  \label{fig:lattice-arith}
\end{figure}

\begin{definition}[Homothetic lattices]
  Two complex lattices $Λ$ and $Λ'$ are said to be \emph{homothetic}
  if there is a complex number $α∈ℂ^{×}$ such that $Λ=αΛ'$.
\end{definition}

Geometrically, applying a homothety to a lattice corresponds to zooms
and rotations around the origin. %
We are only interested in complex tori up to homothety; to classify
them, we introduce the \emph{Eisenstein series of weight $2k$},
defined as
\[G_{2k}(Λ) = \sum_{ω∈Λ\setminus\{0\}}ω^{-2k}.\]
It is customary to set
\[g_2(Λ) = 60G_4(Λ), \quad g_3(Λ) = 140G_6(Λ);\] %
when $Λ$ is clear from the context, we simply write $g_2$ and $g_3$.

\begin{theorem}[Modular $j$-invariant]
  The \emph{modular $j$-invariant} is the function on complex lattices
  defined by
  \[j(Λ) = 1728 \frac{g_2(Λ)^3}{g_2(Λ)^3 - 27g_3(Λ)^2}.\] %
  Two lattices are homothetic if and only if they have the same
  modular $j$-invariant.
\end{theorem}
\begin{proof}
  See~\cite[I, Th.~4.1]{silverman:advanced}.
\end{proof}

It is no chance that the invariants classifying elliptic curves and
complex tori look very similar. %
Indeed, we can prove that the two are in one-to-one correspondence.

\begin{definition}[Weierstrass $℘$ function]
  Let $Λ$ be a complex lattice, the \emph{Weierstrass $℘$ function}
  associated to $Λ$ is the series
  \[℘(z;Λ) = \frac{1}{z^2} + \sum_{ω∈Λ\setminus\{0\}} \left(\frac{1}{(z-ω)^2} - \frac{1}{ω^2}\right).\]
\end{definition}

\begin{theorem}
  \label{th:weierstrass-p}
  The Weierestrass function $℘(z;Λ)$ has the following properties:
  \begin{enumerate}
  \item It is an \emph{elliptic function} for $Λ$, i.e.
    $℘(z) = ℘(z+ω)$ for all $z∈ℂ$ and $ω∈Λ$.
  \item Its Laurent series around $z=0$ is
    \[℘(z) = \frac{1}{z^2} + \sum_{k=1}^∞(2k+1)G_{2k+2}z^{2k}.\]
  \item It satisfies the differential equation
    \[℘'(z)^2 = 4℘(z)^3 - g_2℘(z) - g_3\]
    for all $z∉Λ$.
  \item The curve
    \[E\;:\;y^2=4x^3 - g_2x - g_3\]
    is an elliptic curve over $ℂ$. The map
    \begin{align*}
      ℂ/Λ &\to E(ℂ),\\
      0 &\mapsto (0:1:0),\\
      z &\mapsto (℘(z):℘'(z):1)
    \end{align*}
    is an isomorphism of Riemann surfaces and a group morphism.
  \end{enumerate}
\end{theorem}
\begin{proof}
  See~\cite[VI, Th.~3.1, Th.~3.5, Prop.~3.6]{silverman:elliptic}.
\end{proof}

By comparing the two definitions for the $j$-invariants, we see that
$j(Λ)=j(E)$. %
So, for any homothety class of complex tori, we have a corresponding
isomorphism class of elliptic curves. %
The converse is also true.

\begin{theorem}[Uniformization theorem]
  Let $a,b∈ℂ$ be such that $4a^3+27b^2≠0$, then there is a unique
  complex lattice $Λ$ such that $g_2(Λ) = -4a$ and $g_3(Λ) = -4b$.
\end{theorem}
\begin{proof}
  See~\cite[I, Coro.~4.3]{silverman:advanced}.
\end{proof}

Using the correspondence between elliptic curves and complex tori, we
now have a new perspective on their group structure. %
Looking at complex tori, it becomes immediately evident why the
torsion part has rank $2$, i.e. why $E[m]≃(ℤ/mℤ)^2$. %
This is illustrated in Figure~\ref{fig:torsion}; in the picture we
see two lattices $Λ$ and $Λ'$, generated respectively by the black and
the red dots. %
The multiplication-by-$m$ map corresponds then to
\begin{align*}
  [m] : ℂ/Λ &\to ℂ/Λ',\\
  z &\mapsto z \bmod Λ';
\end{align*}
or equivalently $[m]:z \mapsto mz \bmod Λ$, after applying the
homothety $mΛ'=Λ$, as expected.

\begin{figure}

  \begin{subfigure}{.45\textwidth}
    \centering
    
    \begin{tikzpicture}[scale=1.2]
      \axes{-0.3}{4.5}{-0.5}{4};

      \begin{scope}[/lattice={3}{0.6}{1.2}{2.1}]
        \lattice{-1}{2}

        \foreach \i in {0,...,2} {
          \foreach \j in {0,...,2} {
            \draw[red] (\i/3,\j/3) node[fill,circle,inner sep=1pt] {};
          }
        }
        \draw[red] (0,0) -- (1/3,0) node[yshift=0.2cm] {$a$};
        \draw[red] (0,0) -- (0,1/3) node[yshift=0.2cm] {$b$};

        \draw[blue] (0.8,0.5) node[fill,circle,inner sep=1pt] {}
        node[yshift=0.2cm] {\scriptsize $z$}
        (2/15,1/6) node[fill,circle,inner sep=1pt] {}
        node[yshift=0.2cm] {\scriptsize $3z$};
      \end{scope}
    \end{tikzpicture}  
    \caption{$3$-torsion group on a complex torus (red
      points), with two generators $a$ and $b$, and action of the
      multiplication-by-$3$ map (blue dots).}
    \label{fig:torsion}
  \end{subfigure}
  %%
  \hfill
  %%
  \begin{subfigure}{.45\textwidth}
    \centering
    \begin{tikzpicture}[scale=1.2]
      \axes{-0.3}{4.5}{-0.5}{4};
      
      \begin{scope}[/lattice={3}{0.6}{1.2}{2.1}]
        \lattice{-1}{2}

        \draw[red] (0,0) -- (1/3,0) node[yshift=0.3cm] {$a$};
        \draw[green] (0,0) -- (0,1/3) node[fill,circle,inner sep=1pt] {}
        node[yshift=0.3cm] {$b$};

        \draw[blue] (0.8,0.5) node[fill,circle,inner sep=1pt] {}
        node[yshift=0.3cm] {$z$};
      \end{scope}
      
      \begin{scope}[/lattice={1}{0.2}{1.2}{2.1}]
        \begin{scope}[opacity=0.5,red]
          \lattice[1pt]{-3}{5}
        \end{scope}

        \draw[blue] (0.4,0.5) node[fill,circle,inner sep=1pt] {}
        node[yshift=0.3cm] {$ϕ(z)$};
      \end{scope}
    \end{tikzpicture}
    
    \caption{Isogeny from $ℂ/Λ$ (black dots) to $ℂ/Λ'$ (red dots)
      defined by $ϕ(z)=z \bmod Λ'$. The kernel of $ϕ$ is contained
      in $(ℂ/Λ)[3]$ and is generated by $a$. The kernel of the dual
      isogeny $\hat{ϕ}$ is generated by the vector $b$ in $Λ'$.}
    \label{fig:isogeny}
  \end{subfigure}
  
  \caption{Maps between complex tori.}
\end{figure}

Within this new perspective, isogenies are a mild generalization of
scalar multiplications. %
Whenever two lattices $Λ,Λ'$ verify $αΛ⊂Λ'$, there is a well defined
map
\begin{align*}
   ϕ_α : ℂ/Λ &\to ℂ/Λ',\\
  z &\mapsto αz \bmod Λ'
\end{align*}
that is holomorphic and also a group morphism. %
One example of such maps is given in Figure~\ref{fig:torsion}: there,
$α=1$ and the red lattice strictly contains the black one; the map is
simply defined as reduction modulo $Λ'$. %
It turns out that these maps are exactly the isogenies of the
corresponding elliptic curves.

\begin{theorem}
  Let $E,E'$ be elliptic curves over $ℂ$, with corresponding lattices
  $Λ,Λ'$. %
  There is a bijection between the set of isogenies from $E$ to $E'$
  and the set of maps $ϕ_α$ for all $α$ such that $αΛ⊂Λ'$.
\end{theorem}
\begin{proof}
  See~\cite[VI, Th.~4.1]{silverman:elliptic}.
\end{proof}

Looking again at Figure~\ref{fig:torsion}, we see that there is a
second isogeny $\hat{ϕ}$ from $Λ'$ to $Λ/3$, whose kernel is generated
by $b∈Λ'$. %
The composition $\hat{ϕ}∘ϕ$ is an endomorphism of $ℂ/Λ$, up to the
homothety sending $Λ/3$ to $Λ$, and we verify that it corresponds to
the multiplication-by-$3$ map. %
In this example, the kernels of both $ϕ$ and $\hat{ϕ}$ contain $3$
elements, and we say that $ϕ$ and $\hat{ϕ}$ have \emph{degree} $3$. %
Although not immediately evident from the picture, this same
construction can be applied to any isogeny. %
The isogeny $\hat{ϕ}$ is called the \emph{dual} of $ϕ$. %
Dual isogenies exist not only in characteristic $0$, but also for any
base field, as we shall see in Section~\ref{sec:isogenies}.

Under which conditions does an isogeny become an endomorphism? By
virtue of the last theorem, there is a one-to-one correspondence
between the endomorphisms $E\to E$ and the complex numbers $α$ such
that $αΛ⊂Λ$. %
In general, the only possible choices are given by $α$ an integer,
corresponding to scalar multiplications. %
For some lattices, however, something ``special'' happens; we shall
study this case in Section~\ref{sec:compl-mult}.



\section{Elliptic curves over finite fields}

In this section we shift our attention to elliptic curves defined over
a finite field $k$ with $q$ elements, which are the main objects
manipulated in cryptography. %
Obviously, the group of $k$-rational points is finite, thus the
algebraic group $E(\bar{k})$ only contains torsion elements, and we
have already characterized precisely the structure of the torsion part
of $E$.

For curves over finite fields, the Frobenius endomorphism plays a very
special role, and governs much of their structure.

\begin{definition}[Frobenius endomorphism]
  Let $E$ be an elliptic curve defined over a field with $q$ elements,
  its \emph{Frobenius endomorphism}, denoted by $π$, is the map that
  sends
  \[(X:Y:Z) \mapsto (X^q:Y^q:Z^q).\]
\end{definition}

\begin{proposition}
  \label{th:frob}
  Let $π$ be the Frobenius endomorphism of $E$. Then:
  \begin{itemize}
  \item $\ker π = \{\O\}$;
  \item $\ker (π-1) = E(k)$.
  \end{itemize}
\end{proposition}

\begin{theorem}[Hasse]
  Let $E$ be an elliptic curve defined over a finite field with $q$
  elements. %
  Its Frobenius endomorphism $π$ satisfies a quadratic equation
  \[π^2 - tπ + q = 0,\]
  for some $|t|≤2\sqrt{q}$.
\end{theorem}
\begin{proof}
  See~\cite[V, Th.~2.3.1]{silverman:elliptic}.
\end{proof}

The coefficient $t$ in the equation is called the \emph{trace} of
$π$. %
By replacing $π=1$ in the equation, we immediately obtain the
cardinality of $E$ as $\#E(k) = \#\ker(π-1) = q+1-t$. %

\begin{corollary}
  Let $E$ be an elliptic curve defined over a finite field $k$ with $q$
  elements, then
  \[|\#E(k) - q - 1| ≤ 2\sqrt{q}.\]
\end{corollary}

It turns out that the cardinality of $E$ over its \emph{base field}
$k$ determines its cardinality over any finite extension of it. %
This is a special case of Weil's famous conjectures, proven by Weil
himself in 1949 for Abelian varieties, and more generally by Deligne
in 1973.

\begin{definition}
  Let $V$ be a projective variety defined over a finite field $\F_q$,
  its \emph{zeta function} is the power series
  \[Z(V/\F_q; T) = \exp\left(\sum_{n=1}^∞\#V(\F_{q^n})\frac{T^n}{n}\right).\]
\end{definition}

\begin{theorem}
  \label{th:weil}
  Let $E$ be an elliptic curve defined over a finite field
  $\F_q$, and let $\#E(\F_q)=q+1-a$. Then
  \[Z(E/\F_q;T) = \frac{1-aT+qT^2}{(1-T)(1-qT)}.\]
\end{theorem}
\begin{proof}
  See~\cite[V, Th.~2.4]{silverman:elliptic}.
\end{proof}



\section{Isogenies}
\label{sec:isogenies}

We now look more in detail at isogenies of elliptic curves. %
We start with some basic definitions.

\begin{definition}[Degree, separability]
  Let $ϕ:E\to E'$ be an isogeny defined over a field $k$, and let
  $k(E),k(E')$ be the function fields of $E,E'$. %
  By composing $\phi$ with the functions of $k(E')$, we obtain a
  subfield of $k(E)$ that we denote by $ϕ^\ast(k(E'))$.

  \begin{enumerate}
  \item The \emph{degree} of $ϕ$ is defined as
    $\deg ϕ = [k(E):ϕ^\ast(k(E'))]$; it is always finite.
  \item $ϕ$ is said to be \emph{separable}, \emph{inseparable}, or
    \emph{purely inseparable} if the extension of function fields is.
  \item If $ϕ$ is separable, then $\deg ϕ = \#\ker ϕ$.
  \item If $ϕ$ is purely inseparable, then $\deg ϕ$ is a power of the
    characteristic of $k$.
  \item Any isogeny can be decomposed as a product of a separable and
    a purely inseparable isogeny.
  \end{enumerate}
\end{definition}
\begin{proof}
  See~\cite[II, Th.~2.4]{silverman:elliptic}.
\end{proof}

In practice, most of the time we will be considering separable
isogenies, and we can take $\deg ϕ ≡ \#\ker ϕ$ as the definition of
the degree. %
Notice that in this case $\deg ϕ$ is the size of any fiber of $ϕ$. %

\begin{example}
  The map $ϕ$ from the elliptic curve $y^2=x^3+x$ to $y^2=x^3-4x$
  defined by
  \begin{equation}
    \label{eq:isog-example}
    \begin{aligned}
      ϕ(x,y) &= \left(\frac{x^2+1}{x},y\frac{x^2-1}{x^2}\right),\\
      ϕ(0,0) &= ϕ(\O) = \O
    \end{aligned}
  \end{equation}
  is a separable isogeny between curves defined over $ℚ$. %
  It has degree $2$, and its kernel is generated by the point
  $(0,0)$. %

  \begin{figure}
    \centering
    \begin{tikzpicture}[x=0.03\textwidth,y=0.03\textwidth]
      \begin{scope}
        \node[anchor=center] at (0,7) {$E \;:\; y^2 = x^3 + x$};

        \draw[thin,gray] (0,-6) -- (0,6);
        \draw[thin,gray] (-6,0) -- (6,0);

        \foreach \x/\y in {0/0,5/3,-4/3,-3/5,-2/1,-1/3} {
          \draw[blue,fill] (\x,\y) circle (0.2) node(E_\x_\y){}
          (\x,-\y) circle (0.2) node(E_\x_-\y){};
        }
      \end{scope}

      \draw[black!10!white,thick] (8,-7) -- +(0,14);
      
      \begin{scope}[shift={(16,0)}]
        \node at (0,7) {$E' \;:\; y^2 = x^3 - 4x$};

        \draw[thin,gray] (0,-6) -- (0,6);
        \draw[thin,gray] (-6,0) -- (6,0);

        \foreach \x/\y in {0/0,2/0,3/2,4/2,6/4,-2/0,-1/5} {
          \draw[color=blue,fill] (\x,\y) circle (0.2) node(F_\x_\y){}
          (\x,-\y) circle (0.2) node(F_\x_-\y){};
        }
      \end{scope}

      \begin{scope}[color=red,-latex,dashed]
        \path
        (E_5_3) edge (F_3_2)
        (E_-4_3) edge (F_4_-2)
        (E_-3_5) edge (F_4_2)
        (E_-2_1) edge (F_3_-2)
        (E_-1_3) edge (F_-2_0);
        \path
        (E_5_-3) edge (F_3_-2)
        (E_-4_-3) edge (F_4_2)
        (E_-3_-5) edge (F_4_-2)
        (E_-2_-1) edge (F_3_2)
        (E_-1_-3) edge (F_-2_0);
      \end{scope}
    \end{tikzpicture}
    \caption{The isogeny $(x,y) \mapsto \bigl((x^2+1)/x,\;y(x^2-1)/x^2\bigr)$,
      as a map between curves defined over $\F_{11}$.}
    \label{fig:isog-example}
  \end{figure}

  Plotting the isogeny~\eqref{eq:isog-example} over $ℝ$ would be
  cumbersome, however, since the curves are defined by integer
  coefficients, we may reduce the equations modulo a prime $p$, then
  the isogeny descends to an isogeny of curves over $\F_p$. %
  Figure~\ref{fig:isog-example} plots the action of the isogeny after
  reduction modulo $11$. %
  A red arrow indicates that a point of the left curve is sent onto a
  point on the right curve; the action on the point in $(0,0)$, going to
  the point at infinity, is not shown. %
  We observe a symmetry with respect to the $x$-axis, a consequence of
  the fact that $ϕ$ is a group morphism; and, by looking closer, we may
  also notice that collinear points are sent to collinear points, also a
  necessity for a group morphism. %

  It is evident that the isogeny is $2$-to-$1$, however we are unable to
  see all fibers over $\F_p$, because the isogeny is only surjective
  over the algebraic closure. %
  This is not dissimilar from the way power-by-$n$ maps act on the
  multiplicative group $k^×$ of a field $k$: the map $x↦x^2$, for
  example, is a $2$-to-$1$ (algebraic) group morphism on
  $\F_{11}^\times$, and those elements that have no preimage, the
  non-squares, will have exactly two square roots in $\F_{11^2}$, and so
  on. %
\end{example}

The most unique property of separable isogenies is that they are 
entirely determined by their kernel. %

\begin{proposition}
  \label{prop:isoker}
  Let $E$ be an elliptic curve defined over an algebraically closed
  field, and let $G$ be a finite subgroup of $E$. %
  There is a curve $E'$, and a separable isogeny $ϕ$, such that
  $\ker ϕ=G$ and $ϕ:E→ E'$. %
  Furthermore, $E'$ and $ϕ$ are unique up to composition with an
  isomorphism $E'≃E''$. %
\end{proposition}

Said otherwise, for any finite subgroup $G⊂E$, we have an exact
sequence of algebraic groups
\begin{equation*}
  0 → G → E \overset{ϕ}{→} E' → 0.
\end{equation*}
Uniqueness up to isomorphisms justifies the notation $E/G$ for the
isomorphism class of the image curve $E'$. %
Conversely, since any non-constant morphism of elliptic curves
necessarily has finite kernel, we have a bijection between the finite
subgroups of a curve $E$ and the isogenies with domain $E$ up to
isomorphisms. %
This correspondence is rich in consequences: it is an easy exercise to
prove the following useful facts. %

\begin{corollary}\ 
  \label{coro:isog-basic}
  \begin{enumerate}
  \item Any isogeny of elliptic curves can be decomposed as a product
    of prime degree isogenies.
  \item Let $E$ be defined over an algebraically closed field $k$, let
    $ℓ$ be a prime different from the characteristic of $k$, then
    there are exactly $ℓ+1$ isogenies of degree $ℓ$ with domain $E$,
    up to isomorphism.
  \end{enumerate}
\end{corollary}

Slightly more work is required to prove the following, fundamental,
theorem (the difficulty comes essentially from the inseparable part,
see~\cite[III.6.1]{silverman:elliptic} for a detailed proof).

\begin{theorem}[Dual isogeny theorem]
  Let $ϕ:E→ E'$ be an isogeny of degree $m$. %
  There is a unique isogeny $\hat{ϕ}:E'→ E$ such that
  \[\hat{ϕ}∘ϕ = [m]_E, \quad ϕ∘\hat{ϕ} = [m]_{E'}.\] %
  $\hat{ϕ}$ is called the \emph{dual isogeny of $ϕ$}; it has the
  following properties:
  
  \begin{enumerate}
  \item $\hat{ϕ}$ has degree $m$;
  \item $\hat{ϕ}$ is defined over $k$ if and only if $ϕ$ is;
  \item $\widehat{ψ∘ϕ} = \hat{ϕ}∘\hat{ψ}$ for any isogeny $ψ:E'→ E''$;
  \item $\widehat{ψ+ϕ} = \hat{ψ} + \hat{ϕ}$ for any isogeny $ψ:E→ E'$;
  \item $\deg ϕ = \deg\hat{ϕ}$;
  \item $\hat{\hat{ϕ}} = ϕ$.
  \end{enumerate}
\end{theorem}

The computational counterpart to the kernel-isogeny correspondence is
given by Vélu's much celebrated formulas. %

\begin{proposition}[{Vélu~\cite{velu71}}]
  \label{th:velu}
  Let $E:y^2=x^3+ax+b$ be an elliptic curve defined over a field $k$,
  and let $G⊂E(\bar{k})$ be a finite subgroup. %
  The separable isogeny $ϕ:E→ E/G$, of kernel $G$, can be written as
  \begin{equation*}
    ϕ(P) = \left(
      x(P) + \sum_{Q∈G\setminus\{\O\}}x(P+Q)-x(Q),\\
      y(P) + \sum_{Q∈G\setminus\{\O\}}y(P+Q)-y(Q)
    \right);
  \end{equation*} %
  and the curve $E/G$ has equation $y^2=x^3+a'x+b'$, where
  \begin{align*}
    a' &= a - 5\sum_{Q∈G\setminus\{\O\}}(3x(Q)^2+a),\\
    b' &= b - 7\sum_{Q∈G\setminus\{\O\}}(5x(Q)^3+3ax(Q)+2b).
  \end{align*}
\end{proposition}



\section{Complex multiplication}
\label{sec:compl-mult}

We conclude with one of the most power tools for the study of isogeny
graphs: the theory of \emph{complex multiplication}. %
Our goal is to characterize elliptic curves with endomorphism rings
larger than $ℤ$; to do so, we start from elliptic curves defined over
the complex numbers. %
But first, we need to recall some basic definitions from algebraic
number theory; for a more detailed treatment, see~\cite{langANT}.

An \emph{quadratic number field} is a quadratic extension $K$ of the
rationals; it is called \emph{real} if there exists an embedding
$K⊂ℝ$, \emph{imaginary} otherwise. %
All such fields can be expressed as $ℚ(\sqrt{d})$ for some integer
$d$, the \emph{Gaussian integers} $ℚ(i)$ being a typical example of an
imaginary one. %

\begin{definition}[Discriminant]
  Let $d$ be a square free integer, the \emph{discriminant} of the
  quadratic number field $ℚ(\sqrt{d})$ is $d$ if $d=1\bmod 4$, and
  $4d$ otherwise.
\end{definition}

An integer $Δ$ that is the discriminant of a quadratic number field is
called a \emph{fundamental discriminant}.

\begin{definition}[Ring of integers]
  Let $K$ be a number field, an \emph{algebraic integer} of $K$ is an
  element $α∈K$ that is root of an irreducible monic polynomial with
  integer coefficients. %
  The algebraic integers of $K$ form a ring, called the \emph{ring of
    integers} of $K$.
\end{definition}

For example, $ℤ[i]$ is the ring of integers of $ℚ(i)$; more generally,
if $Δ$ is a fundamental discriminant, the ring of integers of
$ℚ(\sqrt{Δ})$ is $ℤ[δ]$, where $δ=(Δ+\sqrt{Δ})/2$. %
By Definition~\ref{def:order}, an order of a quadratic field $K$ is a
subring of $K$ that is a $ℤ$-module of rank $2$. %
The ring of integers $\O_K$ of $K$ fits the bill: it always has
$(1,δ)$ as \emph{integral basis}, i.e., as a set of $ℤ$-module
generators. %
Furthermore, it is easy to prove that any other order is contained in
$\O_K$; for this reason we will some times call it the \emph{maximal
  order} of $K$. %
More precisely, we can prove the following.

\begin{proposition}
  Let $K$ be a quadratic number field, and let $\O_K$ be its ring of
  integers. %
  Any order $\O⊂K$ can be written as $\O=ℤ+f\O_K$ for an integer $f$,
  called the \emph{conductor} of $\O$. %
  If $Δ_K$ is the discriminant of $K$, the \emph{discriminant} of $\O$
  is $f^2Δ_K$.

  If $\O,\O'$ are two orders of discriminants $Δ,Δ'$, then $\O⊂\O'$ if
  and only if $Δ'|Δ$.
\end{proposition}

When $K$ is imaginary quadratic, any order $\O⊂K⊂ℂ$ is a complex
lattice by definition. %
We now define a broader class of algebraic lattices, that are not
necessarily rings.

\begin{definition}[Fractional ideal]
  Let $\O$ be an order of a number field $K$. %
  A \emph{(fractional) $\O$-ideal} $\a$ is a finitely generated
  non-zero $\O$-submodule of $K$. %
  
  If $\a$ is generated by a single element, then it is called
  \emph{principal}. %
  If $\a⊂\O$, then it is called an \emph{integral} ideal.

  An $\O$-ideal $\a$ is \emph{invertible} if there exists another
  ideal $\a^{-1}$ such that $\a\a^{-1}=\a^{-1}\a=\O$. %
  If $\O$ is the maximal order of $K$, then any $\O$-ideal is
  invertible.
\end{definition}

When $\O$ is the maximal order, we often omit specifying the order,
and simply speak of (fractional) ideals of $K$. %

Now, let $K$ be a quadratic imaginary field. %
Let $Λ$ be a complex lattice such that $Λ⊂K$, and define its order
$\O_Λ$ to be
\begin{equation}
  \label{eq:lattice-order}
  \O_Λ = \{ α ∈ K \;\mid\; αΛ ⊂ Λ \}.
\end{equation}
It is clear that $\O_Λ$ is a ring, and it is easy to show that it is
an order of $K$, and thus that $Λ$ is a fractional $\O_Λ$-ideal. %
Using Theorem~\ref{th:weierstrass-p} we associate to $Λ$ a complex
elliptic curve $E_Λ$; but then, by definition, $\O_Λ≃\End(E_Λ)$. %
Said otherwise, $E_Λ$ \emph{complex multiplication} by $\O_Λ$.

We have thus found a way to construct elliptic curves over the complex
numbers with complex multiplication by a specified order. %
Conversely, every curve with complex multiplication arises this way. %
To show this, we look at the set of all isomorphism classes of
elliptic curves with complex multiplication by a specified order $\O$,
which we will denote by $\Ell(\O)$. %
Because homothetic lattices give rise to isomorphic curves, fractional
ideals $\a$ and $c\a$ will be associated to isomorphic curves $E_\a$
and $E_{c\a}$ as long as $c≠0$. %
This justifies looking at fractional ideals modulo principal ideals.

\begin{definition}[Ideal class group]
  Let $\O$ be an order of a number field $K$. %
  Let $\mathcal{I}(\O)$ be the group of invertible fractional
  $\O$-ideals, and let $\mathcal{P}(\O)$ be the group of principal
  ideals. %

  The \emph{ideal class group} of $\O$ is the quotient group
  \[\Cl(\O) = \mathcal{I}(\O)/\mathcal{P}(\O).\]
  It is a finite Abelian group; its order is called the \emph{class
    number} of $\O$, and denoted by $h(\O)$.
\end{definition}

When $\O$ is the maximal order, $\Cl(\O)$ is also called the class
group of $K$. %
The class group is a fundamental object in \emph{class field theory}:
when $\O$ is the maximal order, it is isomorphic to the Galois group
of the maximal unramified Abelian extension of $K$, also called the
\emph{Hilbert class field} of $K$; more generally, non-maximal orders
are connected to ramified Abelian extensions of $K$. %
The next theorem highlights a fundamental connection between the class
group and the modular $j$-invariant, and thus to elliptic curves with
complex multiplication by $\O$.

\begin{theorem}
  \label{th:compl-mult}
  Let $\O$ be an order of a number field $K$, and let
  $\a_1,\dots,\a_{h(\O)}$ be representatives of $\Cl(\O)$. %
  Then:
  \begin{itemize}
  \item $K(j(\a_i))$ is an Abelian extension of $K$;
  \item The $j(\a_i)$ are all conjugate over $K$;
  \item The Galois group of $K(j(\a_i))$ is isomorphic to $\Cl(\O)$;
  \item $[ℚ(j(\a_i)):ℚ] = [K(j(\a_i)):K] = h(\O)$;
  \item The $j(\a_i)$ are integral, their minimal polynomial is called
    the \emph{Hilbert class polynomial} of $\O$;
  \item $\Cl(\O)$ acts freely and transitively on $\Ell(\O)$, in
    particular $\#\Ell(\O) = h(\O)$.
  \end{itemize}
\end{theorem}
\begin{proof}
  See~\cite[Ch.~II]{silverman:advanced} and~\cite[Ch.~10]{lang1987elliptic}.
\end{proof}

Hence, we have completely characterized all elliptic curves with
complex multiplication by an order $\O$, up to isomorphism; in
particular, we now know that $j$-invariants with complex
multiplication (sometimes called \emph{singular $j$-invariants}) are
algebraic integers. %
In the next part, we shall say more on how $\Cl(\O)$ acts on the set
$\Ell(\O)$.

\begin{example}
  Let $\O=ℤ[i]$, so that $\O$ is the ring of integers of $ℚ(i)$. %
  It was already proven by Gauss that $ℤ[i]$ is a principal ideal
  domain, and thus that its class group is trivial. %
  Up to homothety, there is a unique lattice with order $ℤ[i]$, and
  one such representative is $ℤ[i]$ itself.

  Recall the definition of the Eisenstein series
  \[G_{2k}(Λ) = \sum_{ω∈Λ\setminus\{0\}} ω^{-2k}.\]
  But in our case $Λ=ℤ[i]$, thus $iΛ=Λ$, hence
  \[G_{2k}(Λ) = G_{2k}(iΛ) = i^{-2k}G_{2k}(Λ) = (-1)^kG_{2k}(Λ).\] In
  particular $G_6(Λ) = - G_6(Λ) = 0$, hence, by the definition of the
  modular $j$-invariant, $j(ℤ[i]) = 1728$.

  This shows that that the Hilbert class polynomial of $ℤ[i]$ is
  $X-1728$, and that the curve $E\;:\;y^2=x^3+x$ is the only curve
  over $ℂ$, up to isomorphism, with complex multiplication by
  $ℤ[i]$. %
  In particular, $ℤ[i]$ contains a subgroup of units $\{±1,±i\}$,
  which correspond to the four automorphisms generated by the map
  \begin{align*}
    ι : E &→ E,\\
    (x,y) &↦ (-x,iy).
  \end{align*}
\end{example}


\subsection{Complex multiplication for finite fields}
At this point, we have a complete characterization of complex
multiplication elliptic curves in characteristic $0$. %
What happens, then, in positive characteristic $p$? %

There are at least two ways in which we could construct elliptic
curves over a finite field with endomorphism ring larger than $ℤ$. %
One is to start from a complex multiplication elliptic curve $E$
defined over a number field $L$, and then reduce at a place\footnote{A
  \emph{place} is just a fancy name for a prime ideal of $L$.}
$\frak{p}$ over $p$. %
We write $\bar{E} = E(\frak{p})$ for the reduction of $E$ at the place
$\frak{p}$; if we do this carefully (for example, we must avoid
singular reductions), non-trivial endomorphisms of $E$ will descend to
non-trivial endomorphisms of $\bar{E}$. %

\begin{theorem}[Deuring]
  Let $E$ be an elliptic curve over a number field $L$, with complex
  multiplication by an order $\O⊂K$. %
  Let $\frak{p}$ be a place of $L$ over $p$, and assume that $E$
  has non-singular reduction $\bar{E}$ modulo $\frak{p}$. %
  The curve $\bar{E}$ is supersingular if and only if $p$ has only one
  prime of $K$ above it ($p$ ramifies or remains prime in $k$).

  Suppose that $p$ splits completely in $K$. %
  Let $f$ be the conductor of $\O$, and write $f = p^rf_0$, where
  $p\nmid f_0$. %
  Then:
  \begin{itemize}
  \item $\bar{E}$ has complex multiplication by the order in $K$ with
    conductor $f_0$.
  \item If $p\nmid f$, then the map $ω↦\omega(\frak{p})$ defines an
    isomorphism of $\End(E)$ and $\End(\bar{E})$.
  \end{itemize}
\end{theorem}
\begin{proof}
  See~\cite[Ch.~13]{lang1987elliptic}.
\end{proof}

Note that $p>2$ splits in $K$ if and only if the fundamental
discriminant $Δ_K$ of $K$ is is a square modulo $p$. %
To include the case $p=2$, we may use the Kronecker symbol
$\left(\frac{Δ_K}{p}\right)$, which is equal to $1$ if and only if $p$
splits. %

\begin{example}
  We have seen that the elliptic curve $E/ℚ$ defined by $y^2=x^3+x$
  has complex multiplication by $ℤ[i]$. %
  Assume $p>2$; by virtue of the theorem above, $E(p)$ is
  supersingular if and only if $(-4/p)=-1$, i.e., if and only if
  $p≡3 \bmod 4$.

  In particular, this implies that $-1$ is not a square modulo $p$,
  and thus that the automorphism $(x,y)↦(-x,iy)$ does not descend to
  an $\F_p$-automorphism of $E(p)$. %
  It does, however, descend to an $\F_{p^2}$-automorphism, showing
  that $\End(E(p))$ is not commutative.
\end{example}

Another approach is to directly construct a curve $E/\F_q$ so that its
Frobenius endomorphism is in the desired order. %
Recall that the Frobenius endomorphism $π$ satisfies a quadratic
equation
\[π^2 - tπ + q = 0,\] %
with discriminant $Δ_π=t^2-4q≤0$. %
Setting the case $Δ_π=0$ aside, $\End(E)$ necessarily contains a
subring $ℤ[π]$, isomorphic to an order of $ℚ(\sqrt{Δ_π})$. %
It turns out that these approach is essentially equivalent to the
previous one, as a famous theorem shows.

\begin{theorem}[Deuring's lifting theorem]
  Let $E_0$ be an elliptic curve in characteristic $p$, with an
  endomorphism $ω_o$ which is not trivial. %
  Then there exists an elliptic curve $E$ defined over a number field
  $L$, an endomorphism $ω$ of $E$, and a non-singular reduction of $E$
  at a place $\frak{p}$ of $L$ lying above $p$, such that $E_0$ is
  isomorphic to $E(\frak{p})$, and $ω_0$ corresponds to $ω(\frak{p})$
  under the isomorphism.
\end{theorem}
\begin{proof}
  See~\cite[Ch.~13]{lang1987elliptic}.
\end{proof}


\section*{Exercises}

\begin{exercise}
  Prove Proposition~\ref{th:j}.
\end{exercise}

\begin{exercise}
  Determine all the possible automorphisms of elliptic curves.
\end{exercise}

\begin{exercise}
  Prove Proposition~\ref{th:frob}.
\end{exercise}

\begin{exercise}
  Using Proposition~\ref{th:weil}, devise an algorithm to effectively
  compute $\#E(\F_{q^n})$ given $\#E(\F_q)$.
\end{exercise}

\begin{exercise}
  Prove Corollary~\ref{coro:isog-basic}
\end{exercise}

\begin{exercise}
  Let $K$ be a complex imaginary number field, $Λ⊂K$ a complex
  lattice, and $\O_Λ$ its order as defined in
  Eq.~\eqref{eq:lattice-order}. %
  Prove that $\O_Λ$ is an order of $K$.
\end{exercise}

\begin{exercise}
  Let $ω∈ℂ$ be a cube root of unity, the ring $ℤ[ω]$ is also known as
  the \emph{Eisenstein integers}. %
  Determine all elliptic curves with complex multiplication by $ℤ[ω]$.
\end{exercise}

\begin{exercise}
  Prove that $-163$ is not a square modulo all odd primes
  $<41$. (Hint: $ℚ(\sqrt{-163})$ has class number $1$).
\end{exercise}


%%%%%%%%%%%%%%%%%%%%%%%%%%%%%%%%%%%%%%%%%%%%%%%%%%%%%%

\clearpage
\part{Isogeny graphs}

We now look at isogeny graphs: graphs with isomorphisms classes of
ellitpic curves for vertices, and isogenies for edges. %
Depending on the constraints we put on the isogenies, we will get
graphs with different properties; the most important ones will be
\emph{isogeny volcanoes}, \emph{Cayley graphs}, and
\emph{supersingular graphs}.

The classification of isogeny graphs was initiated by
Mestre~\cite{mestre86}, Pizer~\cite{pizer1,pizer2} and
Kohel~\cite{kohel}; further algorithmic treatment of graphs of
ordinary curves, and the now famous name of \emph{isogeny volcanoes}
was subsequently given by Fouquet and
Morain~\cite{fouquet+morain02}. %
We now review the different kinds of graphs.

\section{Isogeny classes}

We have learned previously that being isogenous is an equivalence
relation,%
\footnote{Reflexivity and transitivity are obvious, symmetry is
  guaranteed by the dual isogeny theorem.} %
it thus makes sens to speak of the \emph{isogeny class} of an elliptic
curve. %
Here, we are interested in characterizing these isogeny classes, and
their connectivity structure. %
We will mostly focus on isogeny classes over finite fields, however we
will occasionally mention the complex case.

We start with a theorem that links isogeny classes with the complex
multiplication theory we previously learned about.

\begin{theorem}[Serre-Tate]
  Two elliptic curves $E,E'$ with complex or quaternionic
  multiplication are isogenous if and only if their \emph{endomorphism
    algebras} $\End(E)⊗ℚ$ and $\End(E')⊗ℚ$ are isomorphic.
\end{theorem}

In layman terms, this theorem is telling us that:
\begin{itemize}
\item Two curves with complex multiplications by $\O$ and $\O'$
  respectively are isogenous if and only if $\O⊂\O'$ or $\O'⊂\O$; or
  equivalently if and only if $\O$ and $\O'$ have the same field of
  fractions.
\item Any two supersingular curves over a field of characteristic $p$
  are isogenous.
\end{itemize}

An easy consequence for the finite field is case is the following.

\begin{corollary}
  \label{coro:serre-tate}
  Two elliptic curves $E,E'$ defined over a finite field $k$ are
  isogenous over $k$ if and only if $\#E(k)=\#E'(k)$.
\end{corollary}

At this stage, we are only interested in elliptic curves up to
isomorphism, i.e., $j$-invariants. %
Accordingly, we say that two $j$-invariants are \emph{isogenous}
whenever their corresponding curves are. %
Like we have already done before, we shall use the notation
$\Ell_q(\O)$ for the subclass of elliptic $j$-invariants over $\bar\F_q$
with complex multiplication by an order $\O$. %

\section{Graphs}

We recall some basic concepts of graph theory; for simplicity, we will
restrict to undirected graphs. %
An undirected graph $G$ is a pair $(V,E)$ where $V$ is a finite set of
\emph{vertices} and $E⊂V×V$ is a set of unordered pairs called
\emph{edges}. %
Two vertices $v,v'$ are said to be \emph{connected by an edge} if
$\{v,v'\}∈E$. %
The \emph{neighbors} of a vertex $v$ are the vertices of $V$ connected
to it by an edge. %
A \emph{path} between two vertices $v,v'$ is a sequence of vertices
$v\to v_1\to\cdots\to v'$ such that each vertex is connected to the
next by an edge. %
The \emph{distance} between two vertices is the length of the shortest
path between them; if there is no such path, the vertices are said to
be at infinite distance. %
A graph is called \emph{connected} if any two vertices have a path
connecting them; it is called \emph{disconnected} otherwise. %
The \emph{diameter} of a connected graph is the largest of all
distances between its vertices. %
The \emph{degree} of a vertex is the number of edges pointing to (or
from) it; a graph where every edge has degree $k$ is called
\emph{$k$-regular}. %
The \emph{adjacency matrix} of a graph $G$ with vertex set
$V=\{v_1,\dots,v_n\}$ and edge set $E$, is the $n×n$ matrix where the
$(i,j)$-th entry is $1$ if there is an edge between $v_i$ and $v_j$,
and $0$ otherwise. %
Because our graphs are undirected, the adjacency matrix is symmetric,
thus it has $n$ real eigenvalues
\[λ_1≥\cdots≥λ_n.\] %

\begin{definition}[Isogeny graph]
  An \emph{isogeny graph} is a (multi)-graph whose vertices are the
  $j$-invariants of isogenous curves, and whose edges are isogenies
  between them.
\end{definition}


\begin{figure}
  \centering
    \begin{tikzpicture}
      \begin{scope}[xshift=6cm]
        \def\crater{7}
        \foreach \i in {1,...,\crater} {
          \draw[fill] (360/\crater*\i:1cm) circle (5pt);
          \draw (360/\crater*\i : 1cm) -- (360/\crater*\i+360/\crater : 1cm);
          \foreach \j in {-1,1} {
            \draw[fill] (360/\crater*\i : 1cm) -- (360/\crater*\i + \j*360/\crater/4 : 2cm) circle (3pt);
            \foreach \k in {-1,0,1} {
              \draw[fill] (360/\crater*\i + \j*360/\crater/4 : 2cm) --
              (360/\crater*\i + + \j*360/\crater/4 + \k*360/\crater/6 : 2.5cm) circle (1pt);
            }
          }
        }
      \end{scope}
      \begin{scope}[xshift=12cm]
        \node at (0,2) {$\End(E)$};
        \draw[fill] (0,1) circle(5pt) node[xshift=0.7cm]{$\O_K$} -- 
        (0,0) circle(3pt) --
        (0,-1) circle(1pt) node[xshift=0.7cm]{$ℤ[π]$};
      \end{scope}
    \end{tikzpicture}
  
    \caption{A volcano of $3$-isogenies (ordinary elliptic curves,
      Elkies case), and the corresponding tower of orders inside the
      endomorphism algebra.}
  \label{fig:volcano}
\end{figure}

The dual isogeny theorem guarantees that for every isogeny $E\to E'$
there is a corresponding isogeny $E'\to E$ of the same degree. %
For this reason, isogeny graphs are usually drawn undirected. %
Figure~\ref{fig:volcano} shows a typical example of isogeny graph over
a finite field, where we restrict to isogenies of degree $3$.


\section{$ℓ$-isogeny graphs}

When we restrict to isogenies of a prescribed degree $ℓ$, we say that
two curves are $ℓ$-isogenous; by the dual isogeny theorem, this is a
symmetric relation. %
Remark that being $ℓ$-isogenous is also well defined up to
isomorphism.

Let us start from the local structure: given an elliptic curve $E$ and
a prime $ℓ$, how many isogenies of degree $ℓ$ have $E$ as domain? %
Thanks to Proposition~\ref{prop:isoker}, we know this is equivalent to
asking how many subgroups of order $ℓ$ the curve has; but then we
immediately know there are exactly $ℓ+1$ isogenies whenever $ℓ≠p$.

For our first example, let us consider a curve $E/ℂ$ \emph{without}
complex multiplication, i.e., such that $\End(E)=ℤ$.  %
Its $ℓ$-isogeny graph, i.e., the connected component of the graph of
$ℓ$-isogenies containing $E$, is $(ℓ+1)$-regular, and cannot have
loops, otherwise that would provide a non-trivial endomorphism of $E$
of degree a power of $ℓ$. %
Hence, the $ℓ$-isogeny graph of $E$ is an infinite $(ℓ+1)$-tree, as
pictured in Figure~\ref{fig:infinite-tree}. %

\begin{figure}
  \centering
    \begin{tikzpicture}[scale=0.6]
      \def\levels{6}
      \draw[fill] (0:0) circle (2pt);
      \foreach \i in {1,...,\levels} {
        \pgfmathparse{3*2^\i}
        \let\nodes\pgfmathresult
        \foreach \j in {1,3,...,\nodes} {
          \pgfmathparse{\j + (-1)^div(\j,2)}
          \let\lower\pgfmathresult
          \ifthenelse{\i = \levels}{
            \draw[dotted] (360/\nodes*\j : \i) --
            (360/\nodes*\lower : \i - 1);
          }{
            \draw[fill] (360/\nodes*\j : \i) circle (2pt) --
            (360/\nodes*\lower : \i - 1);
          }
        }
      }
    \end{tikzpicture}
  
    \caption{Infinite $2$-isogeny graph of elliptic curves without
      complex multiplication.}
  \label{fig:infinite-tree}
\end{figure}

When we think about curves over finite fields, however, some of the
isogenies may only be defined in the algebraic closure, thus we would
like to restrict our graphs to those isogenies that are defined over
$\F_q$. %
Fortunately, we have a Swiss-army-knife to address this question: the
\emph{Frobenius endomorphism} $π$. %
Formally, an isogeny $ϕ:E→E/G$ is $\F_q$-rational if and only if
$π(G)=G$, which suggests looking at the restriction of $π$ to
$E[ℓ]$. %
Assume $ℓ≠p$, then $E[ℓ]$ is a group of rank $2$ and $π$ acts on it
like an element of $\GL_2(\F_ℓ)$, up to conjugation. %
Clearly, the order of $π$ in $\GL_2(\F_ℓ)$ is the degree of the
smallest extension of $\F_q$ where all $ℓ$-isogenies of $E$ are
defined. %
But we can tell even more by diagonalizing the matrix: $π$ must have
between $0$ and $2$ eigenvalues, and the corresponding eigenvectors
define kernels of rational isogenies. %
We thus are in one of the following four cases\footnote{In the point
  counting literature, Case~(0) is known as the \emph{Atkin case}, and
  Case~(2) as the \emph{Elkies case}.}:
\begin{itemize}
\item[(0)] $π$ is not diagonalizable in $\F_ℓ$, then $E$ has no
  $ℓ$-isogenies.
\item[(1.1)] $π$ has one eigenvalue of (geometric) multiplicity one,
  i.e., it is conjugate to a non-diagonal matrix
  $\mat{λ&*\\0&λ}$; then
  $E$ has one $ℓ$-isogeny.
\item[(1.2)] $π$ has one eigenvalue of multiplicity two, i.e., it acts
  like a scalar matrix
  $\mat{λ&0\\0&λ}$; then
  $E$ has $ℓ+1$ isogenies of degree $ℓ$.
\item[(2)] $π$ has two distinct eigenvalues, i.e., it is conjugate to
  a diagonal matrix
  $\mat{λ&0\\0&μ}$ with
  $\lambda\neq\mu$; then $E$ has two $\ell$-isogenies.
\end{itemize}

Naturally, the number of eigenvalues of $π$ depends on the
factorization of its characteristic polynomial $x^2-tx+q$ over $\F_ℓ$,
or equivalently on whether $Δ_π=t^2-4q$ is a square modulo $ℓ$. %

But what about the global structure? %
Any curve $E/\F_q$ can be seen as the reduction modulo $p$ of some
curve $E/\bar{ℚ}$; thus it must inherit the connectivity structure of
the isogeny graph of $E/\bar{ℚ}$. %
However, there is only a finite number of curves defined over $\F_q$,
and not all isogenies will be $\F_q$-rational. %
Thus, the infinite tree must somehow ``fold'' or ``be pruned'' to
fit inside $\F_q$. %

For example, if $E/\F_q$ is a supersingular curve, we shall see later
that its isogeny graph ``folds'' to a finite $(ℓ+1)$-regular graph
containing all supersingular curves, up to $\bar{\F}_q$-isomorphisms.

For the case of ordinary curves, Kohel~\cite{kohel} introduced a
notion of ``depth'' in the graph. %
Let $E/\F_q$ have complex multiplication by an order $\O$ in a number
field $K=ℚ(π)$. %
Write $\O_K$ for the maximal order of $K$, then we know that
$ℤ[π] ⊂ \O ⊂ \O_K$. %
We have already seen that two elliptic curves are isogenous if and
only if they have the same endomorphism algebra $K$; Kohel refined
this as follows.

\begin{proposition}[{Kohel~\cite[Prop.~21]{kohel}}]
  Let $E,E'$ be elliptic curves defined over a finite field, and let
  $\O,\O'$ be their respective endomorphism rings. %
  Suppose that there exists an isogeny $ϕ:E→E'$ of prime degree $ℓ$,
  then $\O$ contains $\O'$ or $\O'$ contains $\O$, and the index of
  one in the other divides $ℓ$.
\end{proposition}

For a fixed prime $ℓ$, Kohel defines a curve $E$ to be \emph{at the
  surface} if $v_ℓ([\O_K:\End(E)])=0$, where $v_ℓ$ is the $ℓ$-adic
valuation. %
$E$ is said to be \emph{at depth $d$} if $v_ℓ([\O_K:\End(E)])=d$; the
maximal depth being $d_{\max}=v_ℓ([\O_K:ℤ[π]])$, curves at depth
$d_{\max}$ are said to be \emph{at the floor (of rationality)}, and
$d_{\max}$ is called the \emph{height} of the graph of $E$. %
Kohel calls then an $ℓ$-isogeny \emph{horizontal} if it goes to a
curve at the same depth, \emph{descending} if it goes to a curve at
greater depth, \emph{ascending} if it goes to a curve at lesser
depth. %

But how many horizontal and vertical $ℓ$-isogenies does a given curve
have?  %
The following theorem gives a complete classification, also summarized
in Table~\ref{tab:periodic-table}. %

\begin{theorem}[{Kohel~\cite{kohel}}]
  \label{prop:isogeny-count}
  Let~$E/\F_q$ be an ordinary elliptic curve, $π$ its Frobenius
  endomorphism, and $Δ_K$ the fundamental discriminant of $ℚ(π)$. %
  \begin{enumerate}
  \item If $E$ is not at the floor, there are $ℓ+1$ isogenies of
    degree $ℓ$ from~$E$, in total.
  \item If $E$ is at the floor, there are no descending $ℓ$-isogenies
    from~$E$.
  \item If $E$ is at the surface, then there are
    $\left(\frac{Δ_K}{ℓ}\right)+1$~horizontal $ℓ$-isogenies from~$E$
    (and no ascending $ℓ$-isogenies).
  \item If $E$ is not at the surface, there are no horizontal
    $ℓ$-isogenies from~$E$, and one ascending $ℓ$-isogeny.
  \end{enumerate}
\end{theorem}
\begin{proof}
  See~\cite[Prop.~21]{kohel}, or~\cite{defeo-hdr}.
\end{proof}

\begin{table}
  \centering
  \def\arraystretch{1.3}
  \begin{tabular}{c | c | c | c c c}
    \multicolumn{3}{c|}{} & \multicolumn{3}{c}{Isogeny types}\\
    \multicolumn{3}{c|}{} & $→$ & $↑$ & $↓$\\
    \hline
    $v_ℓ(Δ_π/Δ_K)=0$ & $ℓ\nmid[\O_K:\O]]$ & $ℓ\nmid[\O:ℤ[π]]$ & $1+\leg{Δ_K}{ℓ}$& &\\
    \hline
    & $ℓ\nmid[\O_K:\O]]$ & $ℓ\mid[\O:ℤ[π]]$ &$1+\leg{Δ_K}{ℓ}$& &$ℓ-\leg{Δ_K}{ℓ}$\\
    $v_ℓ(Δ_π/Δ_K)>1$ & $ℓ\mid[\O_K:\O]]$ & $ℓ\mid[\O:ℤ[π]]$ &  &$1$&$ℓ$\\
    & $ℓ\mid[\O_K:\O]]$ & $ℓ\nmid[\O:ℤ[π]]$ & &$1$& 
  \end{tabular}
  \caption{Number and types of $ℓ$-isogenies, according to splitting
    type of the characteristic polynomial of $π$.}
  \label{tab:periodic-table}
\end{table}

This theorem shows that, away from the surface, isogeny graphs just
look like $ℓ$-regular complete trees of bounded height, with $ℓ$
descending isogenies at every level except the floor. %
However, the surface has a more varied structure:
\begin{itemize}
\item[(0)] If $\leg{Δ_K}{ℓ}=-1$, there are no horizontal isogenies:
  the isogeny graph is just a complete tree of degree $ℓ+1$ (in the
  graph theoretic sense) at each level but the last. %
  We call this the \emph{Atkin case}, as it is an extension of the
  Atkin case in the SEA point counting algorithm.
\item[(1)] If $\leg{Δ_K}{ℓ}=0$, there is exactly one horizontal
  isogeny $ϕ:E→E'$ at the surface. %
  Since $E'$ also has one horizontal isogeny, it necessarily is
  $\hat{ϕ}$, so the surface only contains two elliptic curves, each
  the root of a complete tree. %
  We call this the \emph{ramified case}.
\item[(2)] The case $\leg{Δ_K}{ℓ}=1$ is arguably the most interesting
  one. %
  Each curve at the surface has exactly two horizontal isogenies, thus
  the subgraph made by curves on the surface is two-regular and
  finite, i.e., a cycle. %
  Below each curve of the surface there are $ℓ-1$ curves, each the
  root of a complete tree. %
  We call this the \emph{Elkies case}, again by extension of point
  counting. %
\end{itemize}

\begin{figure}[h]
  \centering
  \begin{tikzpicture}
    \begin{scope}
      \draw[fill] (0,0) circle (2pt)
      -- (-1,-1) circle (2pt)
      (0,0) -- (0,-1) circle (2pt)
      (0,0) -- (1,-1) circle (2pt);
      \node at (0,-2) {Atkin: $\left(\frac{Δ_K}{ℓ}\right) = -1$};
    \end{scope}    

    \begin{scope}[xshift=3.5cm]
      \draw[fill] (0,0) circle (2pt)
      -- (-0.5,-1) circle (2pt)
      (0,0) -- (0.5,-1) circle (2pt)
      (0,0) -- (2,0) circle (2pt)
      -- (1.5,-1) circle (2pt)
      (2,0) -- (2.5,-1) circle (2pt);
      \node at (1,-2) {ramified: $\left(\frac{Δ_K}{ℓ}\right) = 0$};
    \end{scope}
    
    \begin{scope}[xshift=9cm]
      \draw[fill] (-0.8,0) node[coordinate] (A) {} circle (2pt)
      -- +(0,-1) circle (2pt)
      (0,-0.3) node[coordinate] (B) {} circle (2pt)
      -- +(0,-1) circle (2pt)
      (0.8,0) node[coordinate] (C) {} circle (2pt)
      -- +(0,-1) circle (2pt);
      \draw[bend right=20]
      (A) edge (B)
      (B) edge (C)
      (C) edge[dashed,bend right=90] (A);
      \node at (0,-2) {Elkies: $\left(\frac{Δ_K}{ℓ}\right) = +1$};
    \end{scope}
  \end{tikzpicture}
  \caption{The three shapes of volcanoes of $2$-isogenies of height 1.}
  \label{fig:volcanology}
\end{figure}

The three cases are summarized in Figure~\ref{fig:volcanology}. %
Their looks have justified the name if \emph{isogeny volcanoes} for
them~\cite{fouquet+morain02}; in the Elkies case, we call
\emph{crater} the cycle at the surface.

We are left with one last question: how large are these graphs? %
Theorem~\ref{th:compl-mult} tells us that for any order $\O$ there are
exactly $h(\O)$ curves in $\Ell(\O)$, thus we know exactly how many
curves there are in each level of the volcano; for example we know
that there will be exactly $h(\O_K)$ distinct trees in the Atkin
case. %
What we do not know yet, is the number of connected components in the
Elkies case. %
To address this question, we shall go back to complex multiplication.

\section{Complex multiplication}
\label{sec:compl-mult-2}

We have already seen how the theory of complex multiplication gives a
correspondence between the class group $\Cl(\O)$ and the set of CM
elliptic curves $\Ell(\O)$. %
However, the (omitted) proof of Theorem~\ref{th:compl-mult} provides
much more than a simple bijection of sets: it constructs an
\emph{action} of the group $\Cl(\O)$ on the set $\Ell(\O)$. %
We now complete our study of complex multiplication by defining the
group action, and then constructing \emph{Cayley graphs} associated to
it.

We let $\O$ be an order in a number field $K$, and we assume that
$\Ell_q(\O)$ is non-empty. %
Because curves in $\Ell_q(\O)$ are connected exclusively by horizontal
(cyclic) isogenies, we will call it a \emph{horizontal isogeny class}.

Let $E∈\Ell_q(\O)$, let $\a$ be an invertible ideal in $\O$, of norm
coprime to $q$, and define the \emph{${\a}$-torsion} subgroup of $E$
as
\begin{equation*}
  \label{eq:a-torsion}
  E[\a] = \{P ∈ E(\bar{\F}_q) \mid σ(P) = 0
  \text{ for all } σ ∈ \a \}.
\end{equation*}
This subgroup is the kernel of a separable isogeny
$\phi_{\a}:E→E/E[\a]$; it can be proven that $\phi_{\a}$ is
horizontal, and that its degree is the \emph{norm} of $\a$. %
By composing with an appropriate purely inseparable isogeny, the
definition of $ϕ_\a$ is easily extended to invertible ideals of any
norm.

Writing $\a·E$ for the isomorphism class of the image of $ϕ_\a$, we
get an action $·:\mathcal{I}(\O)×\Ell_q(\O)→\Ell_q(\O)$ of the group
of invertible ideals of $\O$ on $\Ell_q(\O)$. %
It is then apparent that endomorphisms of $E$ correspond to principal
ideals in $\O$, and act trivially on $\Ell_q(\O)$. %
Since the action factors through principal ideals, it natural to
consider the induced action of $\Cl(\O)$ on $\Ell_q(\O)$. %
The main theorem of complex multiplication states that this action is
\emph{simply transitive}. %

\begin{theorem}[Complex multiplication]
  Let $\F_q$ be a finite field, $\O⊂ℚ(\sqrt{-D})$ an order in a
  quadratic imaginary field, and $\Ell_q(\O)$ the set of
  $\bar{\F}_q$-isomorphism classes of curves with complex
  multiplication by $\O$. %

  Assume $\Ell_q(\O)$ is non-empty, then it is a \emph{principal
    homogeneous space} for the class group $\Cl(\O)$, under the action
  \begin{align*}
    \Cl(\O) × \Ell_q(\O) &→ \Ell_q(\O),\\
    (\a,E)  &↦ \a·E
  \end{align*}
  defined above.
\end{theorem}

Being a principal homogeneous space (also called a \emph{torsor})
means that, for any fixed base point $E∈\Ell_q(\O)$, there is a
bijection
\[
\begin{aligned}
\Cl(\O) &\longrightarrow \Ell_q(\O) \\
\text{Ideal class of }\a &\longmapsto \text{Isomorphism class of }\a\cdot E.
\end{aligned}
\]
This confirms what we already knew, that $\#\Ell_q(\O)=h(\O)$, but
also answers our question on the size of $ℓ$-isogeny volcanoes.

\begin{corollary}
  Let $\O$ be a quadratic imaginary order, and assume that
  $\Ell_q(\O)$ is non-empty. %
  Let $ℓ$ be a prime such that $\O$ is $ℓ$-maximal, i.e., such that
  $ℓ$ does not divide the conductor of $\O$. %
  All $ℓ$-isogeny volcanoes of curves in $\Ell_q(\O)$ are
  isomorphic. %
  Furthermore, one of the following is true.
  \begin{enumerate}
  \item[(0)] If the ideal $(ℓ)$ is prime in $\O$, then there are
    $h(\O)$ distinct $ℓ$-isogeny volcanoes of Atkin type, with surface
    in $\Ell_q(\O)$.
  \item[(1)] If $(ℓ)$ is ramified in $\O$, i.e., if it decomposes as a
    square $\frak l^2$, then there are $h(\O)/2$ distinct $ℓ$-isogeny
    volcanoes of ramified type, with surface in $\Ell_q(\O)$.
  \item[(2)] If $(ℓ)$ splits as a product $\frak l·\hat{\frak l}$ of
    two distinct prime ideals, then there are $h(\O)/n$ distinct
    $ℓ$-isogeny volcanoes of Elkies type, with craters in $\Ell_q(\O)$
    of size $n$, where $n$ is the order of $\frak l$ in $\Cl(\O)$.
  \end{enumerate}
\end{corollary}

But we can extract even more information from the group action. %
Assume that the Frobenius endomorphism splits modulo $ℓ$, i.e., that
\[π^2 - tπ + q = (π - λ)(π - μ) \mod\ell\] %
for two distinct eigenvalues $λ,μ$. %
Associate to $λ$ and $μ$ the prime ideals $\a=(π-λ,ℓ)$ and
$\hat{\a}=(π-μ,ℓ)$, both of norm $ℓ$; then $E[\a]$ is the eigenspace
of $λ$, and $E[\hat{\a}]$ that of $μ$. %
Because $\a\hat{\a} = \hat{\a}\a = (ℓ)$, the ideal classes $\a$ and
$\hat{\a}$ are the inverse of one another in $\Cl(\O)$, therefore the
isogenies $ϕ_{\a}:E→\a·E$ and $ϕ_{\hat{\a}}:\a·E→E$ are dual to one
another (up to isomorphism). %

Hence, we see that the eigenvalues $λ$ and $μ$ define two opposite
directions on the $\ell$-isogeny crater, independent of the starting
curve, as shown in Figure~\ref{fig:cycle}. %
The size of the crater is the order of $(π-λ,ℓ)$ in $\Cl(\O)$, and the
set $\Ell_q(\O)$ is partitioned into craters of equal size. %
What we have here is a very basic example of \emph{Cayley graph}.

\begin{figure}[t]
  \begin{minipage}{0.45\textwidth}
    \centering
    \begin{tikzpicture}
      \def\crater{7}
      \foreach \i in {1,...,\crater} {
        \begin{scope}[shorten >=0.1cm,->]
          \draw[blue!60!black] (360/\crater*\i : 1.95cm) -- (360/\crater*\i+360/\crater : 1.95cm);
          \draw[blue!60!white] (360/\crater*\i+360/\crater : 2.05cm) -- (360/\crater*\i : 2.05cm);
        \end{scope}
        \draw[blue!60!black] (360/\crater*\i+180/\crater:1.6cm) node {\small$λ$};
        \draw[blue!60!white] (360/\crater*\i+180/\crater:2cm) node {\small$μ$};
      }
      \foreach \i in {1,...,\crater} {
        \draw[fill] (360/\crater*\i:2cm) circle (2pt);
      }
    \end{tikzpicture}
    \caption{An isogeny cycle for an Elkies prime $ℓ$, with edge directions
      associated with the Frobenius eigenvalues $λ$ and $μ$.}
    \label{fig:cycle}
  \end{minipage}
  \hfill
  \begin{minipage}{0.45\textwidth}
    \centering
    \begin{tikzpicture}
      \def\crater{12}
      \def\jumpa{-8}
      \def\jumpb{9}
      \def\diam{2cm}

      \foreach \i in {1,...,\crater} {
        \draw[blue] (360/\crater*\i : \diam) to[bend right] (360/\crater*\i+360/\crater : \diam);
        \draw[red] (360/\crater*\i : \diam) to[bend right] (360/\crater*\i+\jumpa*360/\crater : \diam);
        \draw[green] (360/\crater*\i : \diam) to[bend right=50] (360/\crater*\i+\jumpb*360/\crater : \diam);
      }
      \foreach \i in {1,...,\crater} {
        \pgfmathparse{int(mod(2^\i,13))}
        \let\exp\pgfmathresult
        \draw[fill] (360/\crater*\i: \diam) circle (2pt);% +(360/\crater*\i: 0.4) node{$x^{\exp}$};
      }
    \end{tikzpicture}
    \caption{Graph of horizontal isogenies on 12 curves, with isogenies
      of three different degrees (represented in different colors).}
    \label{fig:cayley}
  \end{minipage}
\end{figure}


\begin{definition}[Cayley graph]
  Let $G$ be a group and $S⊂G$ be a symmetric subset (i.e., $s∈S$
  implies $s^{-1}∈S$). %
  The \emph{Cayley graph} of $(G,S)$ is the undirected graph whose
  vertices are the elements of $G$, and such that there is an edge
  between $g$ and $sg$ if and only if $s∈S$. %
\end{definition}

The graph in Figure~\ref{fig:cycle} is isomorphic to a Cayley graph of
$\Cl(\O)$ for an edge set $S=\{\a,\hat\a\}$, but, unlike the Cayley
graph itself, its vertices are in bijection with $\Cl(\O)$ only up to
automorphism.%
\footnote{Said otherwise, any vertex could correspond to $1$, and we
  do not know which.} %
These kind of graphs obtained from group actions are sometimes called
\emph{Schreier graphs} to distinguish them from the more specific
case.

In the sequel, we shall work with a larger edge set $S$, which will
amount to ``glue many isogeny craters together'', as shown in
Figure~\ref{fig:cayley}.


%%%%%%%%%%%%%%%%%%%%%%%%%%%%%%%%%%%%%%%%%%%%%%%%%%%%%%%%%%%%%%%% 

\section{Quaternionic multiplication}

Supersingular curves are generally not covered
by the theory of complex multiplication. %
For most of them, indeed, the Frobenius endomorphism acts like an element of
$ℤ$, instead of acting like a ``complex multiplier''. %

Supersingular curves are defined by the fact that multiplication by
$p$ is purely inseparable, i.e., $E[p]$ is trivial. %
This implies that the curve $E^{(p^2)}$ is isomorphic to $E$, and thus
that both are isomorphic to a curve defined over $\F_{p^2}$. %

When $E$ is defined over $\F_p$, we can still use what we know about
complex multiplication. %
Indeed, these curves have trace $0$, and thus the Frobenius
endomorphism has two distinct eigenvalue $±\sqrt{-p}$, implying that
$\End_{\F_p}(E)$, the ring of $\F_p$-rational endomorphisms, is
isomorphic to an order in $ℚ(\sqrt{-p})$.

When $E$ is defined over $\F_{p^2}$, however,%
\footnote{This also applies to curves defined over $\F_p$, when we
  extend scalars.} %
its Frobenius endomorphism must satisfy $π^2-tπ+p^2=0$, with $t$ a
multiple of $p$; hence, by Hasse's theorem, $t∈\{0,±p,±2p\}$. %
The cases $t∈\{0,±p\}$ only happen for a very limited number of curves
with $j$-invariant $0$ or $1728$; we are thus mostly interested in the
case $t=±2p$, i.e., $π=±p$. %
Then, the \emph{full} endomorphism ring $\End(E)$ (i.e., not
restricted to $\F_p$-rational endomorphisms) is isomorphic to a
maximal order the quaternion algebra $B_{p,∞}$ ramified at $p$ and at
infinity. %

\begin{example}
  The elliptic curve $y^2=x^3+x$ has supersingular reduction at all
  primes $p=3\bmod 4$. %
  Its ring of $\F_p$-rational endomorphisms is generated by
  $π=\sqrt{-p}$, and it is not maximal in $ℚ(\sqrt{-p})$.

  The automorphism $ι:(x,y)↦(-x,iy)$ is only defined over $\F_{p^2}$,
  and does not commute with $π$. %
  The full endomorphism ring is isomorphic to the order generated by
  $π$ and $ι$ inside the quaternion algebra $B_{p,∞}$.
\end{example}

Like the CM case, isogenies are in correspondence with (left) ideals
of $\O$. %
Unlike the CM case, $B_{p,∞}$ has more than one maximal order, and
there is no concept of \emph{depth}, thus no ascending, descending or
horizontal isogenies. %

More precisely, let $\a⊂Β_{p,∞}$ a lattice, the \emph{left order} of $\a$ is
the ring $\O(\a)=\{x∈B_{p,∞}\mid x\a⊂\a\}$. %
Two lattices $\a,\frak b$ are said to be \emph{right isomorphic} if
$\a=\frak b x$ for some $x∈B_{p,∞}$. %
If $\O⊂B_{p,∞}$ is an order, $\a$ is called a \emph{left ideal} of $\O$ if
$\O⊂\O(\a)$; the \emph{left class set} $\Cl(\O)$ is the set of right
ideal classes of left ideals of $\O$. %
The order $\#\Cl(\O)$ only depends on the quaternion algebra, and is
called the \emph{class number} of $B_{p,∞}$. %
Analogous definitions can be given by swapping left and right; we
refer to~\cite[Chapter~42]{Voight2018} for more properties and
definitions. %

Like in the CM case, the set $\Cl(\O)$ is in bijection with the vertex
set of a supersingular graph. %

\begin{theorem}
  Let $B_{p,∞}$ be the quaternion algebra ramified at $p$ and
  infinity, and let $\O⊂B_{p,∞}$ be a maximal order. %
  Let $E_0/F_{p^2}$ be a supersingular elliptic curve with
  $\End(E_0)≃\O$. %
  \begin{enumerate}
  \item The number of isomorphism classes of supersingular elliptic
    curves is equal to the class number of $B_{p,∞}$.
  \item There is a one-to-one correspondence $\a↦\a·E_0$ between
    $\Cl(\O)$ and the set of isomorphism classes of supersingular
    elliptic curves, such that $\End(\a·E_0)$ is isomorphic to the
    right order of $\a$.
  \end{enumerate}
\end{theorem}

This theorem can be turned into an equivalence of categories,
see~\cite[Theorem~45]{kohel}. %
Thanks to the Eichler mass formula, we obtain the exact size of the
isogeny class. %

\begin{corollary}
  The number of isomorphism classes of supersingular elliptic curves
  is equal to
  \begin{equation*}
    \left\lfloor\frac{p}{12}\right\rfloor +
    \begin{cases}
      0 &\text{if $p=1\mod 12$,}\\
      1 &\text{if $p=5,7\mod 12$,}\\
      2 &\text{if $p=11\mod 12$.}
    \end{cases}
  \end{equation*}
\end{corollary}

We thus have a bound on the size of a supersingular isogeny graph over
$\F_{p^2}$. %
Since the Frobenius acts like a scalar, all isogenies are defined over
$\F_{p^2}$, hence supersingular $ℓ$-isogeny graphs are necessarily
$(ℓ+1)$-regular. %
In the next section we will learn that the supersingular $ℓ$-isogeny
graph has a unique connected component. %


\section{Expander graphs from isogenies}

We are finally introducing two families of isogeny graphs suitable
for cryptographic use. %
We will want them to somehow ``behave like large random graphs'',
while at the same time having a strong algebraic structure: the first
is needed for security, the second to produce complex protocols such
as key exchange. %

The random-like properties of isogeny graphs are typically expressed
in terms of \emph{expansion}. %
An undirected graph on $n$ vertices has $n$ real eigenvalues
$λ_1≥\cdots≥λ_n$, and, if the graph is $k$-regular, it can be proven
that $k=λ_1≥λ_n≥-k$. %
Because of this equality, $λ_1$ is called the \emph{trivial
  eigenvalue}. %
An \emph{expander graph} is a $k$-regular graph such that its
non-trivial eigenvalues are bounded away, in absolute value, from
$k$. %
We recall here some basic facts about expanders; for an in depth
review, see~\cite{Goldreich2011,tao2011expander}.

\begin{definition}[Expander graph]
  Let $ε>0$ and $k≥1$. A $k$-regular graph is called a (one-sided)
  \emph{$ε$-expander} if
  \[λ_2≤(1-ε)k;\]
  and a \emph{two-sided $ε$-expander} if it also satisfies
  \[λ_n≥-(1-ε)k.\] %
  A sequence $G_i=(V_i,E_i)$ of $k$-regular graphs with $\#V_i→∞$ is
  said to be a one-sided (resp. two-sided) \emph{expander family} if
  there is an $ε>0$ such that $G_i$ is a one-sided (resp. two-sided)
  $ε$-expander for all sufficiently large $i$.
\end{definition}

\begin{theorem}[Ramanujan graph]
  Let $k≥1$, and let $G_i$ be a sequence of $k$-regular graphs. %
  Then
  \[\max(|λ_2|,|λ_n|) ≥ 2\sqrt{k-1} - o(1),\]
  as $n→∞$. %
  A graph such that $|λ_j|≤2\sqrt{k-1}$ for any $λ_j$ except $λ_1$ is
  called a \emph{Ramanujan graph}.
\end{theorem}

Two related properties of expander graphs are relevant to us. %
First, they have \emph{short diameter}: as $n→∞$ the diameter of an
expander is bounded by $O(\log n)$, with the constant depending only
on $k$ and $ε$. %
Second, expanders have \emph{rapidly mixing walks}: loosely speaking,
the next proposition says that random walks of length close to the
diameter terminate on any vertex with probability close to uniform. %

\begin{proposition}[Mixing theorem~(\cite{jao+miller+venkatesan09})]
  Let $G=(V,E)$ be a $k$-regular two-sided $ε$-expander. %
  Let $F⊂V$ be any subset of the vertices of $G$, and let $v$ be any
  vertex in $V$. %
  Then a random walk of length at least
  \[\frac{\log(\#F^{1/2}/(2\#V))}{\log(1-ε)}\] %
  starting from $v$ will land in $F$ with probability at least
  $\#F/(2\#V)$.
\end{proposition}

The walk length in the mixing theorem is also called the \emph{mixing
  length} of the expander graph. %

Random regular graphs typically make good expanders, but only a
handful of deterministic constructions is known, most of them based on
Cayley graphs~\cite{LubPS,chung1989diameters,Goldreich2011}. %
We just introduced Cayley graphs constructed from isogeny craters in
Section~\ref{sec:compl-mult-2}, and, unsurprisingly, they turn out to
be expanders, provided we add enough edges to them.

\begin{theorem}[{Jao, Miller, Venkatesan~\cite{jao+miller+venkatesan09}}]
  \label{th:ord-exp}
  Let $\O$ be a quadratic imaginary order, and assume that
  $\Ell_q(\O)$ is non-empty. %
  Let $δ>0$, and define the graph $G$ on $\Ell_q(\O)$ where two
  vertices are connected whenever there is a horizontal isogeny
  between them of prime degree bounded by $O((\log q)^{2+δ})$.

  Then $G$ is a regular graph and, under the generalized Riemann
  hypothesis for the characters of $\Cl(\O)$, there exists an $ε$
  independent of $\O$ and $q$ such that $G$ is a two-sided
  $ε$-expander.
\end{theorem}

The theorem is readily generalized to supersingular
curves and isogenies defined over $\F_p$. %

A radically different construction of expander graphs is given by
graphs of supersingular curves \emph{defined over $\F_{p^2}$} with
$ℓ$-isogenies, for a \emph{single} prime $ℓ≠p$. %
Two examples of such graphs are shown in
Figure~\ref{fig:sup-97-2-3}. %
This construction is related to LPS
graphs~\cite{LubPS,Lub,cryptoeprint:2018:593}, but is not isomorphic
to a Cayley graph. %

\begin{theorem}[{Mestre~\cite{mestre86}, Pizer~\cite{pizer1,pizer2}}]
  \label{th:ss-exp}
  Let $\ell≠p$ be two primes. %
  The $ℓ$-isogeny graph of supersingular curves in $\bar\F_p$, is
  connected, $(ℓ+1)$-regular, and has the Ramanujan property.
\end{theorem}

\begin{figure}
  \centering
  \begin{tikzpicture}
    \def\graph{
      \begin{scope}[every node/.style={fill,black,circle,inner sep=2pt}]
        \node at (0,0)  (1){};
        \node at (0,4) (20){};
        \node at (2,1)  (16z){};
        \node at (-2,1)  (81z){};
        \node at (-1,2) (77z){};
        \node at (1,2)  (20z){};
        \node at (-2,3)  (85z){};
        \node at (2,3)  (12z){};
      \end{scope}
    }
    
    \graph
    \begin{scope}[blue,every loop/.style={looseness=50}]
      \path (1) edge (20) edge (16z) edge (81z);
      \path (20) edge[loop left] (20) edge[loop right] (20);
      \path (16z) edge (81z) edge (77z);
      \path (81z) edge (20z);
      \path (77z) edge (20z) edge (85z);
      \path (20z) edge (12z);
      \path (12z) edge[bend right=10] (85z) edge[bend left=10] (85z);
    \end{scope}
        
    \begin{scope}[xshift=6cm]
      \graph
      \begin{scope}[red]
        \path (1) edge (85z) edge (81z) edge (12z) edge (16z);
        \path (20) edge (85z) edge (77z) edge (20z) edge (12z);
        \path (81z) edge (85z) edge (77z) edge (16z);
        \path (85z) edge (12z);
        \path (12z) edge (16z);
        \path (16z) edge (20z);
        \path (20z) edge[bend right=10] (77z) edge[bend left=10] (77z);
      \end{scope}
    \end{scope}
  \end{tikzpicture}
  \caption{Supersingular isogeny graphs of degree 2 (left, blue) and 3
    (right, red) on $\F_{97^2}$.}
  \label{fig:sup-97-2-3}
\end{figure}


Both of these isogeny graphs will be used in the next part to
build key exchange protocols. %
For reasons that will be apparent soon, there will only be a mild
connection between the expansions properties of the graphs and the
security of the protocols: the expansion theorems will mostly serve as
a blueprint for devising good cryptosystems, but will have no provable
impact. %


\section*{Exercises}

\begin{exercise}
  Prove Corollary~\ref{coro:serre-tate}.
\end{exercise}

\begin{exercise}
  Prove that the dual of a horizontal isogeny is horizontal, and that
  the dual of a descending isogeny is ascending.
\end{exercise}

\begin{exercise}
  Prove that the height of a volcano of $ℓ$-isogenies is $v_ℓ(f_π)$,
  the $ℓ$-adic valuation of the Frobenius endomorphism.
\end{exercise}

\begin{exercise}
  Let $X^2-tX-q$ be the minimal polynomial of $π$, and suppose that it
  splits as $(X-λ)(X-μ)$ in $ℤ_ℓ$ (the ring of $ℓ$-adic integers). %
  Prove that the volcano of $ℓ$ isogenies has height $v_ℓ(λ-μ)$.
\end{exercise}

\begin{exercise}
  \label{ex:group-struct}
  Prove that $E[ℓ]⊂E(\F_q)$ implies $ℓ|(q-1)$.
\end{exercise}

\begin{exercise}
  Find a prime power $q$ and an elliptic curve $E/\F_q$ such that the
  $3$-isogeny volcano of $E$ is the same as the one in
  Figure~\ref{fig:volcano}.
\end{exercise}


%%%%%%%%%%%%%%%%%%%%%%%%%%%%%%%%%%%%%%%%%%%%%%%%%%%%%%

\clearpage
\part{Key exchange}


%%%%%%%%%%%%%%%%%%%%%%%%%%%%%%%%%%%%%%%%%%%%%%%%%%%%%%

\clearpage
\part{Signatures, other protocols, open problems}

%%%%%%%%%%%%%%%%%%%%%%%%%%%%%%%%%%%%%%%%%%%%%%%%%%%%%%
\clearpage
\bibliographystyle{plain}
\bibliography{wurzburg,isogenies_bib/isogenies}

\end{document}

% Local Variables:
% ispell-local-dictionary:"american"
% End:

%  LocalWords:  Isogeny projective affine Abelian invertible isogeny
%  LocalWords:  isomorphism endomorphism endomorphisms isogenies
%  LocalWords:  cardinality automorphisms cryptographic cryptosystem
%  LocalWords:  parallelize homothety Homothetic invariants Expander
%  LocalWords:  supersingular irreducibility cryptographically torsor
%  LocalWords:  expander expanders Schreier coprime quaternion monic
% LocalWords:  morphisms bijection subring morphism surjective
% LocalWords:  discriminants homothetic
