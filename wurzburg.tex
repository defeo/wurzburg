\documentclass[10pt]{article}

\usepackage[a4paper]{geometry}
\usepackage[english]{babel}
\usepackage{array}
\usepackage{amsmath,amsthm,amsfonts,amssymb}
\usepackage{unicode}
\usepackage{subcaption}
\usepackage[type={CC},modifier={by-nc},imagemodifier={-eu},version={4.0},imagewidth=5em]{doclicense}
\usepackage{fancyhdr}

\usepackage{algorithmic}
\renewcommand{\algorithmicrequire}{\textbf{Input:}}
\renewcommand{\algorithmicensure}{\textbf{Output:}}
\algsetup{linenodelimiter=.}

\usepackage[pdfusetitle]{hyperref}
\hypersetup{
  unicode=true,
  colorlinks=true,
  citecolor=blue!70!black,
  filecolor=black,
  linkcolor=red!70!black,
  urlcolor=blue,
  pdfstartview={FitH},
  pdfauthor={Luca De Feo},
  pdfsubject={Mathematics},
  pdfkeywords={Cryptography, Number theory, Elliptic curves, Isogenies},
}

\usepackage{tikz}
\usetikzlibrary{arrows,matrix,decorations,decorations.text,decorations.pathmorphing,calc}
\pgfkeys{/triangle/.code=\tikzset{x={(-0.5cm,-0.866cm)},y={(1cm,0cm)}}}
\pgfkeys{/lattice/.code n args={4}{\tikzset{cm={#1,#2,#3,#4,(0,0)}}}}

\newcommand{\axes}[4]{
  \clip (#1,#3) rectangle (#2,#4);
  \draw [thin, gray, -latex] (#1,0) -- (#2,0);% Draw x axis
  \draw [thin, gray, -latex] (0,#3) -- (0,#4);% Draw y axis
}

\newcommand{\lattice}[3][2pt]{
  \draw[style=help lines,dashed] (#2-1,#2-1) grid[step=1] (#3+1,#3+1);
  \foreach \x in {#2,...,#3}{
    \foreach \y in {#2,...,#3}{
      \node[draw,circle,inner sep=#1,fill] at (\x,\y) {};
      % Places a dot at those points
    }
  }
}

% theorem environments
\theoremstyle{plain}
\newtheorem{theorem}{Theorem}
\newtheorem{lemma}[theorem]{Lemma}
\newtheorem{corollary}[theorem]{Corollary}
\newtheorem{proposition}[theorem]{Proposition}
\theoremstyle{definition}
\newtheorem{definition}[theorem]{Definition}
\newtheorem{example}[theorem]{Example}
\newtheorem{problem}{Problem}
\newtheorem{exercice}{Exercice}[part]

\DeclareMathOperator{\Aut}{Aut}
\DeclareMathOperator{\End}{End} % endomorphism ring
\DeclareMathOperator{\Tr}{Tr} % finite field trace
\DeclareMathOperator{\Gal}{Gal} % Galois group
\DeclareMathOperator{\ord}{ord} % order of an element
\DeclareMathOperator{\lcm}{lcm} % least common multiple
\DeclareMathOperator{\norm}{N} % norm
\DeclareMathOperator{\loglog}{loglog}
\DeclareMathOperator{\im}{Im}
\DeclareMathOperator{\GL}{GL}
\DeclareMathOperator{\SL}{SL}
\DeclareMathOperator{\Cl}{Cl}
\DeclareMathOperator{\Ell}{Ell}

\def\A{\ensuremath{\mathbb{A}}}
\def\P{\ensuremath{\mathbb{P}}}
\def\F{\ensuremath{\mathbb{F}}}
\def\O{\ensuremath{\mathcal{O}}}
\def\tildO{\ensuremath{\tilde{O}}}
\def\euler{\ensuremath{\varphi}}
\def\a{\ensuremath{\mathfrak{a}}}

\newcommand{\bl}[1]{\textcolor{blue}{#1}}
\newcommand{\rd}[1]{\textcolor{red}{#1}}

\title{Isogeny Graphs in Cryptography}
\author{Luca De Feo\\
  Universit\'e Paris Saclay -- UVSQ\\
  \url{https://defeo.lu/}}
\date{Graph Theory Meets Cryptography\\
  July 29 -- August 2, 2019, Wurzb\"urg, Germany}

\begin{document}
\maketitle
\thispagestyle{fancy}
\renewcommand{\headrulewidth}{0pt}
\renewcommand{\footrulewidth}{0.4pt}
\cfoot{\doclicenseThis}
\lfoot{\LaTeX{} source code available at \url{https://github.com/defeo/wurzburg/}.}

\section*{Introduction}

These lectures notes were written for the summer school 
\emph{Graph Theory Meets Cryptography} in Wurzb\"urg, Germany. %

The presentation is divided in four parts, roughly corresponding to
the four lectures given. %

\paragraph{Isogeny Based Cryptography} is a very young field, that has
only begun in the 2000s. %
It has its roots in \emph{Elliptic Curve Cryptography} (ECC), a
somewhat older branch of public-key cryptography that was started in
the 1980s, when Miller and Koblitz first suggested to use elliptic
curves inside the Diffie-Hellman key exchange protocol (see
Section~\ref{sec:appl-diff-hellm}). %

ECC only started to gain traction in the 1990s, after Schoof's
algorithm made it possible to easily find elliptic curves of large
prime order. %
It is nowadays a staple in public-key cryptography. %
The 2000s have seen two major innovations in ECC: the rise of
\emph{Pairing Based Cryptography} (PBC), epitomized by Joux' one-round
tripartite Diffie-Hellman key exchange, and the advent of
Isogeny based cryptography, initiated by the works of Couveignes,
Teske and Rostovtsev \& Stolbunov. %
While PBC has attracted most of the attention during the first decade,
thanks to its revolutionary applications, isogeny based cryptography
has stayed mostly discrete during this time. %
It is only in the second half of the 2010 that the attention has
partly shifted to isogenies. %
The main reason for this is the sudden realization by the
cryptographic community of the very possibly near arrival of a
\emph{general purpose quantum computer}. %
While the capabilities of such futuristic machine would render all of ECC
and PBC suddenly worthless, isogeny based cryptography seems to resist
much better to the cryptanalytic powers of the quantum computer.

In these notes, after a review of the general theory of elliptic
curves and isogenies, we will present the most important isogeny based
systems, and their cryptographic properties.

{
  \hypersetup{linkcolor=black}
  \setcounter{tocdepth}{1}
  \tableofcontents
}

%%%%%%%%%%%%%%%%%%%%%%%%%%%%%%%%%%%%%%%%%%%%%%%%%%%%%%

\clearpage
\part{Elliptic curves and isogenies}

%%%%%%%%%%%%%%%%%%%%%%%%%%%%%%%%%%%%%%%%%%%%%%%%%%%%%%

\clearpage
\part{Isogeny graphs}

%%%%%%%%%%%%%%%%%%%%%%%%%%%%%%%%%%%%%%%%%%%%%%%%%%%%%%

\clearpage
\part{Key exchange}

%%%%%%%%%%%%%%%%%%%%%%%%%%%%%%%%%%%%%%%%%%%%%%%%%%%%%%

\clearpage
\part{Signatures, other protocols, open problems}

%%%%%%%%%%%%%%%%%%%%%%%%%%%%%%%%%%%%%%%%%%%%%%%%%%%%%%
\clearpage
\bibliographystyle{plain}
\bibliography{wurzburg,isogenies_bib/isogenies}

\end{document}

% Local Variables:
% ispell-local-dictionary:"american"
% End:

%  LocalWords:  Isogeny projective affine Abelian invertible isogeny
%  LocalWords:  isomorphism endomorphism endomorphisms isogenies
%  LocalWords:  cardinality automorphisms cryptographic cryptosystem
%  LocalWords:  parallelize homothety Homothetic invariants Expander
%  LocalWords:  supersingular irreducibility cryptographically torsor
%  LocalWords:  expander expanders Schreier coprime quaternion
