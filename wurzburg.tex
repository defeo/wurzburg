\documentclass[10pt]{article}

\usepackage[a4paper]{geometry}
\usepackage[english]{babel}
\usepackage{array}
\usepackage{amsmath,amsthm,amsfonts,amssymb}
\usepackage{unicode}
\usepackage{subcaption}
\usepackage[type={CC},modifier={by-nc},imagemodifier={-eu},version={4.0},imagewidth=5em]{doclicense}
\usepackage{fancyhdr}

\usepackage{algorithmic}
\renewcommand{\algorithmicrequire}{\textbf{Input:}}
\renewcommand{\algorithmicensure}{\textbf{Output:}}
\algsetup{linenodelimiter=.}

\usepackage[pdfusetitle]{hyperref}
\hypersetup{
  unicode=true,
  colorlinks=true,
  citecolor=blue!70!black,
  filecolor=black,
  linkcolor=red!70!black,
  urlcolor=blue,
  pdfstartview={FitH},
  pdfauthor={Luca De Feo},
  pdfsubject={Mathematics},
  pdfkeywords={Cryptography, Number theory, Elliptic curves, Isogenies},
}

\usepackage{tikz}
\usetikzlibrary{arrows,matrix,decorations,decorations.text,decorations.pathmorphing,calc}
\pgfkeys{/triangle/.code=\tikzset{x={(-0.5cm,-0.866cm)},y={(1cm,0cm)}}}
\pgfkeys{/lattice/.code n args={4}{\tikzset{cm={#1,#2,#3,#4,(0,0)}}}}

\newcommand{\axes}[4]{
  \clip (#1,#3) rectangle (#2,#4);
  \draw [thin, gray, -latex] (#1,0) -- (#2,0);% Draw x axis
  \draw [thin, gray, -latex] (0,#3) -- (0,#4);% Draw y axis
}

\newcommand{\lattice}[3][2pt]{
  \draw[style=help lines,dashed] (#2-1,#2-1) grid[step=1] (#3+1,#3+1);
  \foreach \x in {#2,...,#3}{
    \foreach \y in {#2,...,#3}{
      \node[draw,circle,inner sep=#1,fill] at (\x,\y) {};
      % Places a dot at those points
    }
  }
}

% theorem environments
\theoremstyle{plain}
\newtheorem{theorem}{Theorem}
\newtheorem{lemma}[theorem]{Lemma}
\newtheorem{corollary}[theorem]{Corollary}
\newtheorem{proposition}[theorem]{Proposition}
\theoremstyle{definition}
\newtheorem{definition}[theorem]{Definition}
\newtheorem{example}[theorem]{Example}
\newtheorem{problem}{Problem}
\newtheorem{exercise}{Exercise}[part]

\DeclareMathOperator{\Aut}{Aut}
\DeclareMathOperator{\End}{End} % endomorphism ring
\DeclareMathOperator{\Tr}{Tr} % finite field trace
\DeclareMathOperator{\Gal}{Gal} % Galois group
\DeclareMathOperator{\ord}{ord} % order of an element
\DeclareMathOperator{\lcm}{lcm} % least common multiple
\DeclareMathOperator{\norm}{N} % norm
\DeclareMathOperator{\loglog}{loglog}
\DeclareMathOperator{\im}{Im}
\DeclareMathOperator{\GL}{GL}
\DeclareMathOperator{\SL}{SL}
\DeclareMathOperator{\Cl}{Cl}
\DeclareMathOperator{\Ell}{Ell}

\def\A{\ensuremath{\mathbb{A}}}
\def\P{\ensuremath{\mathbb{P}}}
\def\F{\ensuremath{\mathbb{F}}}
\def\O{\ensuremath{\mathcal{O}}}
\def\tildO{\ensuremath{\tilde{O}}}
\def\euler{\ensuremath{\varphi}}
\def\a{\ensuremath{\mathfrak{a}}}

\newcommand{\bl}[1]{\textcolor{blue}{#1}}
\newcommand{\rd}[1]{\textcolor{red}{#1}}

\title{Isogeny Graphs in Cryptography}
\author{Luca De Feo\\
  Universit\'e Paris Saclay -- UVSQ\\
  \url{https://defeo.lu/}}
\date{Graph Theory Meets Cryptography\\
  July 29 -- August 2, 2019, Wurzb\"urg, Germany}

\begin{document}
\maketitle
\thispagestyle{fancy}
\renewcommand{\headrulewidth}{0pt}
\renewcommand{\footrulewidth}{0.4pt}
\cfoot{\doclicenseThis}
\lfoot{\LaTeX{} source code available at \url{https://github.com/defeo/wurzburg/}.}

\section*{Introduction}

These lectures notes were written for the summer school 
\emph{Graph Theory Meets Cryptography} in Wurzb\"urg, Germany. %

The presentation is divided in four parts, roughly corresponding to
the four lectures given. %

{
  \hypersetup{linkcolor=black}
  \setcounter{tocdepth}{1}
  \tableofcontents
}

%%%%%%%%%%%%%%%%%%%%%%%%%%%%%%%%%%%%%%%%%%%%%%%%%%%%%%

\clearpage
\part{Elliptic curves and isogenies}

In this part, we review the basic and not-so-basic theory of elliptic
curves. %
Our goal is to summarize the fundamental theorems necessary to
understanding the foundations of isogeny based cryptography. %
A proper treatment of the material covered here would require more
than one book, we thus skip proofs and lots of details to go straight
to the useful theorems. %
The reader in search of a more comprehensive treatment will find more
details~\cite{silverman:elliptic,silverman:advanced,lang1987elliptic,neukirch2013algebraic}. %

Throughout this part we let $k$ be a field, and we denote by $\bar{k}$
its algebraic closure. %

\section{Elliptic curves}

Elliptic curves are projective curves of genus 1 with a distinguished
point. %
Projective space initially appeared through the process of adding
\emph{points at infinity}, as a method to understand the geometry of
projections (also known as \emph{perspective} in classical
painting). %
In modern terms, we define projective space as the collection of all
lines in affine space passing through the origin.

\begin{definition}[Projective space]
  The \emph{projective space of dimension $n$}, denoted by $\P^n$ or
  $\P^n(\bar{k})$, is the set of all $(n+1)$-tuples
  \[(x_0,\dots,x_n) ∈ \bar{k}^{n+1}\] %
  such that $(x_0,\dots,x_n) ≠ (0,\dots,0)$, taken modulo the
  equivalence relation
  \[(x_0,\dots,x_n) \sim (y_0,\dots,y_n)\] %
  if and only if there exists $λ\in\bar{k}$ such that
  $x_i=λ_iy_i$ for all $i$.
\end{definition}

The equivalence class of a projective point $(x_0,\dots,x_n)$ is
customarily denoted by $(x_0:\cdots:x_n)$. %
The set of the \emph{$k$-rational points}, denoted by $\P^n(k)$, is
defined as
\[\P^n(k) = \left\{(x_0:\cdots:x_n)∈\P^n\;\middle|\; x_i ∈ k \text{ for all $i$}\right\}.\] %
By fixing arbitrarily the coordinate $x_n=0$, we define a projective
space of dimension $n-1$, which we call the \emph{hyperplane at
  infinity}; its points are called \emph{points at infinity}.

From now on we suppose that the field $k$ has characteristic different
from $2$ and $3$. %
This has the merit of greatly simplifying the representation of an
elliptic curve. %
For a general definition, see~\cite[Chap.~III]{silverman:elliptic}.

\begin{definition}[Weierstrass equation]
  An \emph{elliptic curve} defined over $k$ is the locus in
  $\P^2(\bar{k})$ of an equation
  \begin{equation}
    \label{eq:weierstrass}
    Y^2Z = X^3 + aXZ^2 + bZ^3,    
  \end{equation}
  with $a,b∈k$ and $4a^3+27b^2\ne0$.

  The point $(0:1:0)$ is the only point on the line $Z=0$; it is
  called the \emph{point at infinity} of the curve.
\end{definition}

It is customary to write Eq.~\eqref{eq:weierstrass} in \emph{affine
  form}. %
By defining the coordinates $x=X/Z$ and $y=Y/Z$, we equivalently
define the elliptic curve as the locus of the equation
\[y^2 = x^3 + ax +b,\]
plus the point at infinity $\O=(0:1:0)$.

In characteristic different from $2$ and $3$, we can show that any
projective curve of genus $1$ with a distinguished point $\O$ is
isomorphic to a Weierstrass equation by sending $\O$ onto the point at
infinity $(0:1:0)$.

\begin{figure}
  \centering
  \hfill
  %% 
  \begin{tikzpicture}[domain=-2.4566:4,samples=100,yscale=3/8,xscale=3/4]
    \draw plot (\x,{sqrt(\x*\x*\x-4*\x+5)});
    \draw plot (\x,{-sqrt(\x*\x*\x-4*\x+5)});

    \draw[thin,gray,-latex] (0,-7) -- (0,7);
    \draw[thin,gray,-latex] (-3,0) -- (4,0);

    \draw (-3,1) -- (4,8/3+3);
    \begin{scope}[every node/.style={draw,circle,inner sep=1pt,fill},cm={1,2/3,0,0,(0,3)}]
      \node at (-2.287980,0) {};
      \node at (-0.535051,0) {};
      \node at (3.267475,0) {};
    \end{scope}
    \begin{scope}[every node/.style={yshift=0.3cm},cm={1,2/3,0,0,(0,3)}]
      \node at (-2.287980,0) {$P$};
      \node at (-0.535051,0) {$Q$};
      \node at (3.267475,0) {$R$};
    \end{scope}

    \draw[dashed] (3.267475,3.267475*2/3+3) -- (3.267475,-3.267475*2/3-3) 
    node[draw,circle,inner sep=1pt,fill] {}
    node[xshift=-0.1cm,anchor=east] {$P+Q$};
  \end{tikzpicture}
  %% 
  \hfill
  %% 
  \begin{tikzpicture}[domain=-2.4566:4,samples=100,yscale=3/8,xscale=3/4]
    \draw plot (\x,{sqrt(\x*\x*\x-4*\x+5)});
    \draw plot (\x,{-sqrt(\x*\x*\x-4*\x+5)});

    \draw[thin,gray,-latex] (0,-7) -- (0,7);
    \draw[thin,gray,-latex] (-3,0) -- (4,0);
    
    \def\c{3.269524}
    \def\P{-1.398674}
    \def\R{2.908459}
    \draw (-3,-1+\c) -- (4,4/3+\c);
    \begin{scope}[every node/.style={draw,circle,inner sep=1pt,fill},cm={1,1/3,0,0,(0,3.269524)}]
      \node at (\P,0) {};
      \node at (\R,0) {};
    \end{scope}
    \begin{scope}[every node/.style={yshift=0.3cm},cm={1,1/3,0,0,(0,3.269524)}]
      \node at (\P,0) {$P$};
      \node at (\R,0) {$R$};
    \end{scope}

    \draw[dashed] (\R,\R/3+\c) -- (\R,-\R/3-\c) 
    node[draw,circle,inner sep=1pt,fill] {}
    node[xshift=-0.1cm,anchor=east] {$[2]P$};
  \end{tikzpicture}
  %%
  \hfill
  \strut
  
  \caption{An elliptic curve defined over $ℝ$, and the geometric
    representation of its group law.}
  \label{fig:weierstrass}
\end{figure}

Now, since any elliptic curve is defined by a cubic equation, Bezout's
theorem tells us that any line in $\P^2$ intersects the curve in
exactly three points, taken with multiplicity. %
We define a group law by requiring that three co-linear points sum to
zero. %

\begin{definition}
  Let $E\;:\;y^2=x^3+ax+b$ be an elliptic curve. Let $P_1=(x_1,y_1)$
  and $P_2=(x_2,y_2)$ be two points on $E$ different from the point at
  infinity, then we define a composition law $⊕$ on $E$ as
  follows:
  \begin{itemize}
  \item $P ⊕ \O = \O ⊕ P = P$ for any point $P∈E$;
  \item If $x_1=x_2$ and $y_1=-y_2$, then $P_1⊕P_2 = \O$;
  \item Otherwise set
    \[λ =
      \begin{cases}
        \frac{y_2-y_1}{x_2-x_1} &\text{if $P≠Q$,}\\
        \frac{3x_1^2+a}{2y_1} &\text{if $P=Q$,}
      \end{cases}
    \]
    then the point $(P_1⊕P_2)=(x_3,y_3)$ is defined by
    \begin{align*}
      x_3 &= λ^2 - x_1 - x_2,\\
      y_3 &= -λx_3 - y_1 + λx_1.
    \end{align*}
  \end{itemize}
\end{definition}

It can be shown that the above law defines an Abelian group, thus we
will simply write $+$ for $⊕$. %
The $n$-th scalar multiple of a point $P$ will be denoted by $[n]P$. %
When $E$ is defined over $k$, the subgroup of its \emph{rational
  points over $k$} is customarily denoted $E(k)$. %
Figure~\ref{fig:weierstrass} shows a graphical depiction of the group
law on an elliptic curve defined over $ℝ$.

We now turn to the group structure of elliptic curves. %
The torsion part is easily characterized.

\begin{proposition}
  Let $E$ be an elliptic curve defined over an algebraically closed
  field $k$, and let $m≠0$ be an integer. %
  The $m$-torsion group of $E$, denoted by $E[m]$, has the following
  structure:
  \begin{itemize}
  \item $E[m] ≃ (ℤ/mℤ)^2$ if the characteristic of $k$ does not divide
    $m$;
  \item If $p>0$ is the characteristic of $k$, then 
    \[E[p^i] ≃
      \begin{cases}
        ℤ/p^iℤ & \text{for any $i≥0$, or}\\
        \{\O\} & \text{for any $i≥0$.}
      \end{cases}
    \]
  \end{itemize}
\end{proposition}
\begin{proof}
  See~\cite[Coro.~6.4]{silverman:elliptic}. For the characteristic $0$
  case see also Section~\ref{sec:elliptic-curves-over}.
\end{proof}

For curves defined over a field of positive characteristic $p$, the
case $E[p]≃ℤ/pℤ$ is called \emph{ordinary}, while the case
$E[p]≃\{\O\}$ is called \emph{supersingular}. %
We shall see an alternative characterization of supersingularity in
the next section.

The free part of the group is much harder to characterize. %
We have some partial results for elliptic curves over number fields.

\begin{theorem}[Mordell-Weil]
  Let $k$ be a number field, the group $E(k)$ is finitely generated.
\end{theorem}

However the exact determination of the rank of $E(k)$ is somewhat
elusive: we have algorithms to compute the rank of most elliptic
curves over number fields; however, an exact formula for such rank is
the object of the
\href{https://en.wikipedia.org/wiki/Birch_and_Swinnerton-Dyer_conjecture}{\it
  Birch and Swinnerton-Dyer conjecture}, one of the
\href{https://en.wikipedia.org/wiki/Millennium_Prize_Problems}{\it
  Clay Millenium Prize Problems}.

\section{Maps between elliptic curves}

Finally, we focus on maps between elliptic curves. %
We are mostly interested in maps that preserve both facets of elliptic
curves: as projective varieties, and as groups. %

We first look into invertible algebraic maps, that is linear changes
of coordinates that preserve the Weierstrass form of the equation. %
Because linear maps preserve lines, it is immediate that they also
preserve the group law. %
It is easily verified that the only such maps take the form
\[(x,y) \mapsto (u^2x', u^3y')\] %
for some $u∈\bar{k}$, thus defining an \emph{isomorphism} between the
curve $y^2=x^3+au^4x+bu^6$ and the curve $(y')^2 = (x')^3 + ax' +
b$. %
Isomorphism classes are traditionally encoded by an invariant, whose
origins can be traced back to complex analysis.

\begin{proposition}[$j$-invariant]
  \label{th:j}
  Let $E:y^2=x^3+ax+b$ be an elliptic curve, and define the
  \emph{$j$-invariant} of $E$ as
  \[j(E) = 1728\frac{4a^3}{4a^3+27b^2}.\] %
  Two curves are isomorphic over the algebraic closure $\bar{k}$ if
  and only if they have the same $j$-invariant.
\end{proposition}

Note that if two curves defined over $k$ are isomorphic over
$\bar{k}$, they are so over an extension of $k$ of degree dividing
$6$. %
An isomorphism between two elliptic curves defined over $k$, that is
itself not defined over $k$ is called a \emph{twist}. %
Any curve has a \emph{quadratic twist}, unique up to isomorphism,
obtained by taking $u∉k$ such that $u^2∈k$. %
The two curves of $j$-invariant $0$ and $1728$ also have \emph{cubic},
\emph{sextic} and \emph{quartic twists}.

A surjective group morphism, not necessarily invertible, between two
elliptic curves is called an \emph{isogeny}. %
It turns out that isogenies are algebraic maps as well.

\begin{theorem}
  Let $E,E'$ be two elliptic curves, and let $\phi:E→E$ be a map between
  them. %
  The following conditions are equivalent:
  \begin{enumerate}
  \item $\phi$ is a surjective group morphism,
  \item $\phi$ is a group morphism with finite kernel,
  \item $\phi$ is a non-constant algebraic map of projective varieties
    sending the point at infinity of $E$ onto the point at infinity of
    $E'$.
  \end{enumerate}
\end{theorem}
\begin{proof}
  See~\cite[III, Th.~4.8]{silverman:elliptic}.
\end{proof}

Two curves are called \emph{isogenous} if there exists an isogeny
between them. %
We shall see later that this is an equivalence relation.

Isogenies from a curve to itself are called \emph{endomorphisms}. %
The prototypical endomorphism is the multiplication-by-$m$
endomorphism defined by
\[[m]\;:\; P \mapsto [m]P.\] %
Its kernel is exactly the $m$-th torsion subgroup $E[m]$. %

Since they are algebraic group morphisms, we can define addition of
isogenies by $(ϕ+ψ)(P) = ϕ(P)+ψ(P)$, and the resulting map is still an
isogeny. %
Thus, by including the constant map that sends every point to the
point at infinity, the set of isogenies $E\to E'$ forms a group. %
Additionally, endomorphisms $E\to E$ support composition, distributing
over addition, hence the set of all endomorphisms forms a ring,
denoted by $\End(E)$.%
\footnote{In short, isogenies are the morphisms in the Abelian
  category of elliptic curves.}

Since each $m∈ℤ$ defines a different multiplication-by-$m$
endomorphism, clearly $ℤ⊂\End(E)$. %
But can $\End(E)$ be larger? %
We shall now give a complete characterization of the endomorphism ring
for any elliptic curve.

\begin{definition}[Order]
  \label{def:order}
  Let $K$ be a finitely generated $ℚ$-algebra. %
  An \emph{order} $\O⊂K$ is a subring of $K$ that is a finitely
  generated $ℤ$-module, and that contains a $ℚ$-basis for $K$.
\end{definition}

The prototypical example of order is the ring of integers $\O_K$ of a
number field $K$. %
It turns out that $\O_K$ is the \emph{maximal order} of $K$, i.e., it
contains any other order of $K$. %
We shall discuss this case in depth in Section~\ref{sec:compl-mult}.

\begin{definition}[Quaternion algebra]
  A \emph{quaternion algebra} is an algebra of the form
  \[K = ℚ + αℚ + βℚ + αβℚ,\]
  where the generators satisfy the relations
  \[α^2,β^2∈ℚ, \quad α^2<0, \quad β^2 < 0, \quad βα=-αβ.\]
\end{definition}

\begin{theorem}[Deuring]
  Let $E$ be an elliptic curve defined over a field $k$ of
  characteristic $p$. %
  The ring $\End(E)$ is isomorphic to one of the following:
  \begin{itemize}
  \item $ℤ$, only if $p=0$;
  \item An order $\O$ in a quadratic imaginary field (a number field
    of the form $ℚ(\sqrt{-D})$ for some $D>0$); in this case we say
    that $E$ has \emph{complex multiplication} by $\O$;
  \item Only if $p>0$, a maximal order in a quaternion algebra
    ramified at $p$ and $∞$; in this case we say that $E$ is
    \emph{supersingular}.
  \end{itemize}
\end{theorem}
\begin{proof}
  See~\cite[III, Coro.~9.4]{silverman:elliptic}
  and~\cite{kohel}.
\end{proof}

In positive characteristic, a curve that is not supersingular is
called \emph{ordinary}; we shall see that it necessarily has complex
multiplication. %



\section{Elliptic curves over $ℂ$}
\label{sec:elliptic-curves-over}

To better understand elliptic curves and their morphisms, we take a
moment now to specialize them to the complex numbers.

\begin{definition}[Complex lattice]
  A \emph{complex lattice} $Λ$ is a discrete subgroup of $ℂ$ that
  contains an $ℝ$-basis of $ℂ$.
\end{definition}

Explicitly, a complex lattice is generated by a \emph{basis}
$(ω_1,ω_2)$, such that $ω_1≠λω_2$ for any $λ∈ℝ$, as
\[Λ = ω_1ℤ + ω_2ℤ.\] %
Up to exchanging $ω_1$ and $ω_2$, we can assume that $\im(ω_1/ω_2)>0$;
we then say that the basis has \emph{positive orientation}. %
A positively oriented basis is obviously not unique, though.

\begin{proposition}
  \label{th:basis-change}
  Let $Λ$ be a complex lattice, and let $(ω_1,ω_2)$ be a positively
  oriented basis, then any other positively oriented basis
  $(ω_1',ω_2')$ is of the form
  \begin{align*}
    ω_1' &= aω_1 + bω_2,\\
    ω_1' &= cω_1 + dω_2,
  \end{align*}
  for some matrix
  $\left(\begin{smallmatrix}a&b\\c&d\end{smallmatrix}\right)∈\SL_2(ℤ)$.
\end{proposition}
\begin{proof}
  See~\cite[I, Lem.~2.4]{silverman:advanced}.
\end{proof}

\begin{definition}[Complex torus]
  Let $Λ$ be a complex lattice, the quotient $ℂ/Λ$ is called a
  \emph{complex torus}.
\end{definition}

\begin{figure}
  \centering
  \begin{tikzpicture}[scale=2]
    \axes{-1}{3.5}{-0.5}{3}

    \begin{scope}[/lattice={1}{0.2}{0.4}{0.7}]
      \draw[fill,black!10] (0,0) -- (1,0) -- (1,1) -- (0,1) -- (0,0);
      \node at (0.5,0.5) {$ℂ/Λ$};
      \node at (0.9,-0.1) {$ω_2$};
      \node at (-0.1,0.9) {$ω_1$};

      \lattice{-3}{4}
    \end{scope}  
  \end{tikzpicture}

  \caption{A complex lattice (black dots) and its associated complex
    torus (grayed \emph{fundamental domain}).}
  \label{fig:lattice}
\end{figure}

A convex set of class representatives of $ℂ/Λ$ is called a
\emph{fundamental parallelogram}. %
Figure~\ref{fig:lattice} shows a complex lattice generated by a
(positively oriented) basis $(ω_1,ω_2)$, together with a fundamental
parallelogram for $ℂ/(ω_1,ω_2)$. %
The additive group structure of $ℂ$ carries over to $ℂ/Λ$, and can be
graphically represented as operations on points inside a fundamental
parallelogram. %
This is illustrated in Figure~\ref{fig:lattice-arith}.

\begin{figure}
  \centering
  \begin{tikzpicture}[scale=1.8]
    \axes{-0.5}{3.5}{-0.5}{3}

    \begin{scope}[/lattice={1}{0.2}{0.4}{0.7}]
      \lattice{-3}{4}

      \node[red] at (0.7,0.65) {$a$}; 
      \node[draw,circle,inner sep=1pt,fill,red] at (0.8,0.5) {};
      \node[red] at (0.2,0.9) {$b$}; 
      \node[draw,circle,inner sep=1pt,fill,red] at (0.3,0.7) {};
      
      \node[draw,circle,inner sep=1pt,fill,red] at (1.1,1.2) {};

      \draw[red,thin,dotted] (0,0) -- (0.8,0.5) -- (1.1,1.2)
      (0,0) -- (0.3,0.7) -- (1.1,1.2);          

      \node[red] at (0.2,0.3) {$a+b$}; 
      \node[draw,circle,inner sep=1pt,fill,red] at (0.1,0.2) {};
    \end{scope}  
  \end{tikzpicture}
  %%
  \hfill
  %%
  \begin{tikzpicture}[scale=1.8]
    \axes{-0.5}{3.5}{-0.5}{3}

    \begin{scope}[/lattice={1}{0.2}{0.4}{0.7}]
      \lattice{-3}{4}
      
      \node[red,yshift=0.2cm] at (0.8,0.6) {$a$}; 
      \draw[red] (0.8,0.6) node[fill,circle,inner sep=1pt] {};

      \draw[red,dotted] (0,0) -- (1.6,1.2) node[fill,circle,inner sep=1pt] {} 
      -- (2.4,1.8) node[fill,circle,inner sep=1pt] {};

      \node[red,yshift=0.3cm] at (0.4,0.8) {$[3]a$}; 
      \draw[red] (0.4,0.8) node[fill,circle,inner sep=1pt] {};
    \end{scope}
  \end{tikzpicture}
  \caption{Addition (left) and scalar multiplication (right) of points
    in a complex torus $ℂ/Λ$.}
  \label{fig:lattice-arith}
\end{figure}

\begin{definition}[Homothetic lattices]
  Two complex lattices $Λ$ and $Λ'$ are said to be \emph{homothetic}
  if there is a complex number $α∈ℂ^{×}$ such that $Λ=αΛ'$.
\end{definition}

Geometrically, applying a homothety to a lattice corresponds to zooms
and rotations around the origin. %
We are only interested in complex tori up to homothety; to classify
them, we introduce the \emph{Eisenstein series of weight $2k$},
defined as
\[G_{2k}(Λ) = \sum_{ω∈Λ\setminus\{0\}}ω^{-2k}.\]
It is customary to set
\[g_2(Λ) = 60G_4(Λ), \quad g_3(Λ) = 140G_6(Λ);\] %
when $Λ$ is clear from the context, we simply write $g_2$ and $g_3$.

\begin{theorem}[Modular $j$-invariant]
  The \emph{modular $j$-invariant} is the function on complex lattices
  defined by
  \[j(Λ) = 1728 \frac{g_2(Λ)^3}{g_2(Λ)^3 - 27g_3(Λ)^2}.\] %
  Two lattices are homothetic if and only if they have the same
  modular $j$-invariant.
\end{theorem}
\begin{proof}
  See~\cite[I, Th.~4.1]{silverman:advanced}.
\end{proof}

It is no chance that the invariants classifying elliptic curves and
complex tori look very similar. %
Indeed, we can prove that the two are in one-to-one correspondence.

\begin{definition}[Weierstrass $℘$ function]
  Let $Λ$ be a complex lattice, the \emph{Weierstrass $℘$ function}
  associated to $Λ$ is the series
  \[℘(z;Λ) = \frac{1}{z^2} + \sum_{ω∈Λ\setminus\{0\}} \left(\frac{1}{(z-ω)^2} - \frac{1}{ω^2}\right).\]
\end{definition}

\begin{theorem}
  \label{th:weierstrass-p}
  The Weierestrass function $℘(z;Λ)$ has the following properties:
  \begin{enumerate}
  \item It is an \emph{elliptic function} for $Λ$, i.e.
    $℘(z) = ℘(z+ω)$ for all $z∈ℂ$ and $ω∈Λ$.
  \item Its Laurent series around $z=0$ is
    \[℘(z) = \frac{1}{z^2} + \sum_{k=1}^∞(2k+1)G_{2k+2}z^{2k}.\]
  \item It satisfies the differential equation
    \[℘'(z)^2 = 4℘(z)^3 - g_2℘(z) - g_3\]
    for all $z∉Λ$.
  \item The curve
    \[E\;:\;y^2=4x^3 - g_2x - g_3\]
    is an elliptic curve over $ℂ$. The map
    \begin{align*}
      ℂ/Λ &\to E(ℂ),\\
      0 &\mapsto (0:1:0),\\
      z &\mapsto (℘(z):℘'(z):1)
    \end{align*}
    is an isomorphism of Riemann surfaces and a group morphism.
  \end{enumerate}
\end{theorem}
\begin{proof}
  See~\cite[VI, Th.~3.1, Th.~3.5, Prop.~3.6]{silverman:elliptic}.
\end{proof}

By comparing the two definitions for the $j$-invariants, we see that
$j(Λ)=j(E)$. %
So, for any homothety class of complex tori, we have a corresponding
isomorphism class of elliptic curves. %
The converse is also true.

\begin{theorem}[Uniformization theorem]
  Let $a,b∈ℂ$ be such that $4a^3+27b^2≠0$, then there is a unique
  complex lattice $Λ$ such that $g_2(Λ) = -4a$ and $g_3(Λ) = -4b$.
\end{theorem}
\begin{proof}
  See~\cite[I, Coro.~4.3]{silverman:advanced}.
\end{proof}

Using the correspondence between elliptic curves and complex tori, we
now have a new perspective on their group structure. %
Looking at complex tori, it becomes immediately evident why the
torsion part has rank $2$, i.e. why $E[m]≃(ℤ/mℤ)^2$. %
This is illustrated in Figure~\ref{fig:torsion}; in the picture we
see two lattices $Λ$ and $Λ'$, generated respectively by the black and
the red dots. %
The multiplication-by-$m$ map corresponds then to
\begin{align*}
  [m] : ℂ/Λ &\to ℂ/Λ',\\
  z &\mapsto z \bmod Λ';
\end{align*}
or equivalently $[m]:z \mapsto mz \bmod Λ$, after applying the
homothety $mΛ'=Λ$, as expected.

\begin{figure}

  \begin{subfigure}{.45\textwidth}
    \centering
    
    \begin{tikzpicture}[scale=1.2]
      \axes{-0.3}{4.5}{-0.5}{4};

      \begin{scope}[/lattice={3}{0.6}{1.2}{2.1}]
        \lattice{-1}{2}

        \foreach \i in {0,...,2} {
          \foreach \j in {0,...,2} {
            \draw[red] (\i/3,\j/3) node[fill,circle,inner sep=1pt] {};
          }
        }
        \draw[red] (0,0) -- (1/3,0) node[yshift=0.2cm] {$a$};
        \draw[red] (0,0) -- (0,1/3) node[yshift=0.2cm] {$b$};

        \draw[blue] (0.8,0.5) node[fill,circle,inner sep=1pt] {}
        node[yshift=0.2cm] {\scriptsize $z$}
        (2/15,1/6) node[fill,circle,inner sep=1pt] {}
        node[yshift=0.2cm] {\scriptsize $3z$};
      \end{scope}
    \end{tikzpicture}  
    \caption{$3$-torsion group on a complex torus (red
      points), with two generators $a$ and $b$, and action of the
      multiplication-by-$3$ map (blue dots).}
    \label{fig:torsion}
  \end{subfigure}
  %%
  \hfill
  %%
  \begin{subfigure}{.45\textwidth}
    \centering
    \begin{tikzpicture}[scale=1.2]
      \axes{-0.3}{4.5}{-0.5}{4};
      
      \begin{scope}[/lattice={3}{0.6}{1.2}{2.1}]
        \lattice{-1}{2}

        \draw[red] (0,0) -- (1/3,0) node[yshift=0.3cm] {$a$};
        \draw[green] (0,0) -- (0,1/3) node[fill,circle,inner sep=1pt] {}
        node[yshift=0.3cm] {$b$};

        \draw[blue] (0.8,0.5) node[fill,circle,inner sep=1pt] {}
        node[yshift=0.3cm] {$z$};
      \end{scope}
      
      \begin{scope}[/lattice={1}{0.2}{1.2}{2.1}]
        \begin{scope}[opacity=0.5,red]
          \lattice[1pt]{-3}{5}
        \end{scope}

        \draw[blue] (0.4,0.5) node[fill,circle,inner sep=1pt] {}
        node[yshift=0.3cm] {$ϕ(z)$};
      \end{scope}
    \end{tikzpicture}
    
    \caption{Isogeny from $ℂ/Λ$ (black dots) to $ℂ/Λ'$ (red dots)
      defined by $ϕ(z)=z \bmod Λ'$. The kernel of $ϕ$ is contained
      in $(ℂ/Λ)[3]$ and is generated by $a$. The kernel of the dual
      isogeny $\hat{ϕ}$ is generated by the vector $b$ in $Λ'$.}
    \label{fig:isogeny}
  \end{subfigure}
  
  \caption{Maps between complex tori.}
\end{figure}

Within this new perspective, isogenies are a mild generalization of
scalar multiplications. %
Whenever two lattices $Λ,Λ'$ verify $αΛ⊂Λ'$, there is a well defined
map
\begin{align*}
   ϕ_α : ℂ/Λ &\to ℂ/Λ',\\
  z &\mapsto αz \bmod Λ'
\end{align*}
that is holomorphic and also a group morphism. %
One example of such maps is given in Figure~\ref{fig:torsion}: there,
$α=1$ and the red lattice strictly contains the black one; the map is
simply defined as reduction modulo $Λ'$. %
It turns out that these maps are exactly the isogenies of the
corresponding elliptic curves.

\begin{theorem}
  Let $E,E'$ be elliptic curves over $ℂ$, with corresponding lattices
  $Λ,Λ'$. %
  There is a bijection between the set of isogenies from $E$ to $E'$
  and the set of maps $ϕ_α$ for all $α$ such that $αΛ⊂Λ'$.
\end{theorem}
\begin{proof}
  See~\cite[VI, Th.~4.1]{silverman:elliptic}.
\end{proof}

Looking again at Figure~\ref{fig:torsion}, we see that there is a
second isogeny $\hat{ϕ}$ from $Λ'$ to $Λ/3$, whose kernel is generated
by $b∈Λ'$. %
The composition $\hat{ϕ}∘ϕ$ is an endomorphism of $ℂ/Λ$, up to the
homothety sending $Λ/3$ to $Λ$, and we verify that it corresponds to
the multiplication-by-$3$ map. %
In this example, the kernels of both $ϕ$ and $\hat{ϕ}$ contain $3$
elements, and we say that $ϕ$ and $\hat{ϕ}$ have \emph{degree} $3$. %
Although not immediately evident from the picture, this same
construction can be applied to any isogeny. %
The isogeny $\hat{ϕ}$ is called the \emph{dual} of $ϕ$. %
Dual isogenies exist not only in characteristic $0$, but also for any
base field, as we shall see in Section~\ref{sec:isogenies}.

Under which conditions does an isogeny become an endomorphism? By
virtue of the last theorem, there is a one-to-one correspondence
between the endomorphisms $E\to E$ and the complex numbers $α$ such
that $αΛ⊂Λ$. %
In general, the only possible choices are given by $α$ an integer,
corresponding to scalar multiplications. %
For some lattices, however, something ``special'' happens; we shall
study this case in Section~\ref{sec:compl-mult}.



\section{Elliptic curves over finite fields}

In this section we shift our attention to elliptic curves defined over
a finite field $k$ with $q$ elements, which are the main objects
manipulated in cryptography. %
Obviously, the group of $k$-rational points is finite, thus the
algebraic group $E(\bar{k})$ only contains torsion elements, and we
have already characterized precisely the structure of the torsion part
of $E$.

For curves over finite fields, the Frobenius endomorphism plays a very
special role, and governs much of their structure.

\begin{definition}[Frobenius endomorphism]
  Let $E$ be an elliptic curve defined over a field with $q$ elements,
  its \emph{Frobenius endomorphism}, denoted by $π$, is the map that
  sends
  \[(X:Y:Z) \mapsto (X^q:Y^q:Z^q).\]
\end{definition}

\begin{proposition}
  \label{th:frob}
  Let $π$ be the Frobenius endomorphism of $E$. Then:
  \begin{itemize}
  \item $\ker π = \{\O\}$;
  \item $\ker (π-1) = E(k)$.
  \end{itemize}
\end{proposition}

\begin{theorem}[Hasse]
  Let $E$ be an elliptic curve defined over a finite field with $q$
  elements. %
  Its Frobenius endomorphism $π$ satisfies a quadratic equation
  \[π^2 - tπ + q = 0,\]
  for some $|t|≤2\sqrt{q}$.
\end{theorem}
\begin{proof}
  See~\cite[V, Th.~2.3.1]{silverman:elliptic}.
\end{proof}

The coefficient $t$ in the equation is called the \emph{trace} of
$π$. %
By replacing $π=1$ in the equation, we immediately obtain the
cardinality of $E$ as $\#E(k) = \#\ker(π-1) = q+1-t$. %

\begin{corollary}
  Let $E$ be an elliptic curve defined over a finite field $k$ with $q$
  elements, then
  \[|\#E(k) - q - 1| ≤ 2\sqrt{q}.\]
\end{corollary}

It turns out that the cardinality of $E$ over its \emph{base field}
$k$ determines its cardinality over any finite extension of it. %
This is a special case of Weil's famous conjectures, proven by Weil
himself in 1949 for Abelian varieties, and more generally by Deligne
in 1973.

\begin{definition}
  Let $V$ be a projective variety defined over a finite field $\F_q$,
  its \emph{zeta function} is the power series
  \[Z(V/\F_q; T) = \exp\left(\sum_{n=1}^∞\#V(\F_{q^n})\frac{T^n}{n}\right).\]
\end{definition}

\begin{theorem}
  \label{th:weil}
  Let $E$ be an elliptic curve defined over a finite field
  $\F_q$, and let $\#E(\F_q)=q+1-a$. Then
  \[Z(E/\F_q;T) = \frac{1-aT+qT^2}{(1-T)(1-qT)}.\]
\end{theorem}
\begin{proof}
  See~\cite[V, Th.~2.4]{silverman:elliptic}.
\end{proof}



\section{Isogenies}
\label{sec:isogenies}

We now look more in detail at isogenies of elliptic curves. %
We start with some basic definitions.

\begin{definition}[Degree, separability]
  Let $ϕ:E\to E'$ be an isogeny defined over a field $k$, and let
  $k(E),k(E')$ be the function fields of $E,E'$. %
  By composing $\phi$ with the functions of $k(E')$, we obtain a
  subfield of $k(E)$ that we denote by $ϕ^\ast(k(E'))$.

  \begin{enumerate}
  \item The \emph{degree} of $ϕ$ is defined as
    $\deg ϕ = [k(E):ϕ^\ast(k(E'))]$; it is always finite.
  \item $ϕ$ is said to be \emph{separable}, \emph{inseparable}, or
    \emph{purely inseparable} if the extension of function fields is.
  \item If $ϕ$ is separable, then $\deg ϕ = \#\ker ϕ$.
  \item If $ϕ$ is purely inseparable, then $\deg ϕ$ is a power of the
    characteristic of $k$.
  \item Any isogeny can be decomposed as a product of a separable and
    a purely inseparable isogeny.
  \end{enumerate}
\end{definition}
\begin{proof}
  See~\cite[II, Th.~2.4]{silverman:elliptic}.
\end{proof}

In practice, most of the time we will be considering separable
isogenies, and we can take $\deg ϕ ≡ \#\ker ϕ$ as the definition of
the degree. %
Notice that in this case $\deg ϕ$ is the size of any fiber of $ϕ$. %

\begin{example}
  The map $ϕ$ from the elliptic curve $y^2=x^3+x$ to $y^2=x^3-4x$
  defined by
  \begin{equation}
    \label{eq:isog-example}
    \begin{aligned}
      ϕ(x,y) &= \left(\frac{x^2+1}{x},y\frac{x^2-1}{x^2}\right),\\
      ϕ(0,0) &= ϕ(\O) = \O
    \end{aligned}
  \end{equation}
  is a separable isogeny between curves defined over $ℚ$. %
  It has degree $2$, and its kernel is generated by the point
  $(0,0)$. %

  \begin{figure}
    \centering
    \begin{tikzpicture}[x=0.03\textwidth,y=0.03\textwidth]
      \begin{scope}
        \node[anchor=center] at (0,7) {$E \;:\; y^2 = x^3 + x$};

        \draw[thin,gray] (0,-6) -- (0,6);
        \draw[thin,gray] (-6,0) -- (6,0);

        \foreach \x/\y in {0/0,5/3,-4/3,-3/5,-2/1,-1/3} {
          \draw[blue,fill] (\x,\y) circle (0.2) node(E_\x_\y){}
          (\x,-\y) circle (0.2) node(E_\x_-\y){};
        }
      \end{scope}

      \draw[black!10!white,thick] (8,-7) -- +(0,14);
      
      \begin{scope}[shift={(16,0)}]
        \node at (0,7) {$E' \;:\; y^2 = x^3 - 4x$};

        \draw[thin,gray] (0,-6) -- (0,6);
        \draw[thin,gray] (-6,0) -- (6,0);

        \foreach \x/\y in {0/0,2/0,3/2,4/2,6/4,-2/0,-1/5} {
          \draw[color=blue,fill] (\x,\y) circle (0.2) node(F_\x_\y){}
          (\x,-\y) circle (0.2) node(F_\x_-\y){};
        }
      \end{scope}

      \begin{scope}[color=red,-latex,dashed]
        \path
        (E_5_3) edge (F_3_2)
        (E_-4_3) edge (F_4_-2)
        (E_-3_5) edge (F_4_2)
        (E_-2_1) edge (F_3_-2)
        (E_-1_3) edge (F_-2_0);
        \path
        (E_5_-3) edge (F_3_-2)
        (E_-4_-3) edge (F_4_2)
        (E_-3_-5) edge (F_4_-2)
        (E_-2_-1) edge (F_3_2)
        (E_-1_-3) edge (F_-2_0);
      \end{scope}
    \end{tikzpicture}
    \caption{The isogeny $(x,y) \mapsto \bigl((x^2+1)/x,\;y(x^2-1)/x^2\bigr)$,
      as a map between curves defined over $\F_{11}$.}
    \label{fig:isog-example}
  \end{figure}

  Plotting the isogeny~\eqref{eq:isog-example} over $ℝ$ would be
  cumbersome, however, since the curves are defined by integer
  coefficients, we may reduce the equations modulo a prime $p$, then
  the isogeny descends to an isogeny of curves over $\F_p$. %
  Figure~\ref{fig:isog-example} plots the action of the isogeny after
  reduction modulo $11$. %
  A red arrow indicates that a point of the left curve is sent onto a
  point on the right curve; the action on the point in $(0,0)$, going to
  the point at infinity, is not shown. %
  We observe a symmetry with respect to the $x$-axis, a consequence of
  the fact that $ϕ$ is a group morphism; and, by looking closer, we may
  also notice that collinear points are sent to collinear points, also a
  necessity for a group morphism. %

  It is evident that the isogeny is $2$-to-$1$, however we are unable to
  see all fibers over $\F_p$, because the isogeny is only surjective
  over the algebraic closure. %
  This is not dissimilar from the way power-by-$n$ maps act on the
  multiplicative group $k^×$ of a field $k$: the map $x↦x^2$, for
  example, is a $2$-to-$1$ (algebraic) group morphism on
  $\F_{11}^\times$, and those elements that have no preimage, the
  non-squares, will have exactly two square roots in $\F_{11^2}$, and so
  on. %
\end{example}

The most unique property of separable isogenies is that they are 
entirely determined by their kernel. %

\begin{proposition}
  Let $E$ be an elliptic curve defined over an algebraically closed
  field, and let $G$ be a finite subgroup of $E$. %
  There is a curve $E'$, and a separable isogeny $ϕ$, such that
  $\ker ϕ=G$ and $ϕ:E→ E'$. %
  Furthermore, $E'$ and $ϕ$ are unique up to composition with an
  isomorphism $E'≃E''$. %
\end{proposition}

Said otherwise, for any finite subgroup $G⊂E$, we have an exact
sequence of algebraic groups
\begin{equation*}
  0 → G → E \overset{ϕ}{→} E' → 0.
\end{equation*}
Uniqueness up to isomorphisms justifies the notation $E/G$ for the
isomorphism class of the image curve $E'$. %
Conversely, since any non-constant morphism of elliptic curves
necessarily has finite kernel, we have a bijection between the finite
subgroups of a curve $E$ and the isogenies with domain $E$ up to
isomorphisms. %
This correspondence is rich in consequences: it is an easy exercise to
prove the following useful facts. %

\begin{corollary}\ 
  \label{coro:isog-basic}
  \begin{enumerate}
  \item Any isogeny of elliptic curves can be decomposed as a product
    of prime degree isogenies.
  \item Let $E$ be defined over an algebraically closed field $k$, let
    $ℓ$ be a prime different from the characteristic of $k$, then
    there are exactly $ℓ+1$ isogenies of degree $ℓ$ with domain $E$,
    up to isomorphism.
  \end{enumerate}
\end{corollary}

Slightly more work is required to prove the following, fundamental,
theorem (the difficulty comes essentially from the inseparable part,
see~\cite[III.6.1]{silverman:elliptic} for a detailed proof).

\begin{theorem}[Dual isogeny theorem]
  Let $ϕ:E→ E'$ be an isogeny of degree $m$. %
  There is a unique isogeny $\hat{ϕ}:E'→ E$ such that
  \[\hat{ϕ}∘ϕ = [m]_E, \quad ϕ∘\hat{ϕ} = [m]_{E'}.\] %
  $\hat{ϕ}$ is called the \emph{dual isogeny of $ϕ$}; it has the
  following properties:
  
  \begin{enumerate}
  \item $\hat{ϕ}$ has degree $m$;
  \item $\hat{ϕ}$ is defined over $k$ if and only if $ϕ$ is;
  \item $\widehat{ψ∘ϕ} = \hat{ϕ}∘\hat{ψ}$ for any isogeny $ψ:E'→ E''$;
  \item $\widehat{ψ+ϕ} = \hat{ψ} + \hat{ϕ}$ for any isogeny $ψ:E→ E'$;
  \item $\deg ϕ = \deg\hat{ϕ}$;
  \item $\hat{\hat{ϕ}} = ϕ$.
  \end{enumerate}
\end{theorem}

The computational counterpart to the kernel-isogeny correspondence is
given by Vélu's much celebrated formulas. %

\begin{proposition}[{Vélu~\cite{velu71}}]
  \label{th:velu}
  Let $E:y^2=x^3+ax+b$ be an elliptic curve defined over a field $k$,
  and let $G⊂E(\bar{k})$ be a finite subgroup. %
  The separable isogeny $ϕ:E→ E/G$, of kernel $G$, can be written as
  \begin{equation*}
    ϕ(P) = \left(
      x(P) + \sum_{Q∈G\setminus\{\O\}}x(P+Q)-x(Q),\\
      y(P) + \sum_{Q∈G\setminus\{\O\}}y(P+Q)-y(Q)
    \right);
  \end{equation*} %
  and the curve $E/G$ has equation $y^2=x^3+a'x+b'$, where
  \begin{align*}
    a' &= a - 5\sum_{Q∈G\setminus\{\O\}}(3x(Q)^2+a),\\
    b' &= b - 7\sum_{Q∈G\setminus\{\O\}}(5x(Q)^3+3ax(Q)+2b).
  \end{align*}
\end{proposition}



\section{Complex multiplication}
\label{sec:compl-mult}

We conclude with one of the most power tools for the study of isogeny
graphs: the theory of \emph{complex multiplication}. %
Our goal is to characterize elliptic curves with endomorphism rings
larger than $ℤ$; to do so, we start from elliptic curves defined over
the complex numbers. %
But first, we need to recall some basic definitions from algebraic
number theory; for a more detailed treatment, see~\cite{langANT}.

An \emph{quadratic number field} is a quadratic extension $K$ of the
rationals; it is called \emph{real} if there exists an embedding
$K⊂ℝ$, \emph{imaginary} otherwise. %
All such fields can be expressed as $ℚ(\sqrt{d})$ for some integer
$d$, the \emph{Gaussian integers} $ℚ(i)$ being a typical example of an
imaginary one. %

\begin{definition}[Discriminant]
  Let $d$ be a square free integer, the \emph{discriminant} of the
  quadratic number field $ℚ(\sqrt{d})$ is $d$ if $d=1\bmod 4$, and
  $4d$ otherwise.
\end{definition}

An integer $Δ$ that is the discriminant of a quadratic number field is
called a \emph{fundamental discriminant}.

\begin{definition}[Ring of integers]
  Let $K$ be a number field, an \emph{algebraic integer} of $K$ is an
  element $α∈K$ that is root of an irreducible monic polynomial with
  integer coefficients. %
  The algebraic integers of $K$ form a ring, called the \emph{ring of
    integers} of $K$.
\end{definition}

For example, $ℤ[i]$ is the ring of integers of $ℚ(i)$; more generally,
if $Δ$ is a fundamental discriminant, the ring of integers of
$ℚ(\sqrt{Δ})$ is $ℤ[δ]$, where $δ=(Δ+\sqrt{Δ})/2$. %
By Definition~\ref{def:order}, an order of a quadratic field $K$ is a
subring of $K$ that is a $ℤ$-module of rank $2$. %
The ring of integers $\O_K$ of $K$ fits the bill: it always has
$(1,δ)$ as \emph{integral basis}, i.e., as a set of $ℤ$-module
generators. %
Furthermore, it is easy to prove that any other order is contained in
$\O_K$; for this reason we will some times call it the \emph{maximal
  order} of $K$. %
More precisely, we can prove the following.

\begin{proposition}
  Let $K$ be a quadratic number field, and let $\O_K$ be its ring of
  integers. %
  Any order $\O⊂K$ can be written as $\O=ℤ+f\O_K$ for an integer $f$,
  called the \emph{conductor} of $\O$. %
  If $Δ_K$ is the discriminant of $K$, the \emph{discriminant} of $\O$
  is $f^2Δ_K$.

  If $\O,\O'$ are two orders of discriminants $Δ,Δ'$, then $\O⊂\O'$ if
  and only if $Δ'|Δ$.
\end{proposition}

When $K$ is imaginary quadratic, any order $\O⊂K⊂ℂ$ is a complex
lattice by definition. %
We now define a broader class of algebraic lattices, that are not
necessarily rings.

\begin{definition}[Fractional ideal]
  Let $\O$ be an order of a number field $K$. %
  A \emph{(fractional) $\O$-ideal} $\a$ is a finitely generated
  non-zero $\O$-submodule of $K$. %
  
  If $\a$ is generated by a single element, then it is called
  \emph{principal}. %
  If $\a⊂\O$, then it is called an \emph{integral} ideal.

  An $\O$-ideal $\a$ is \emph{invertible} if there exists another
  ideal $\a^{-1}$ such that $\a\a^{-1}=\a^{-1}\a=\O$. %
  If $\O$ is the maximal order of $K$, then any $\O$-ideal is
  invertible.
\end{definition}

When $\O$ is the maximal order, we often omit specifying the order,
and simply speak of (fractional) ideals of $K$. %

Now, let $K$ be a quadratic imaginary field. %
Let $Λ$ be a complex lattice such that $Λ⊂K$, and define its order
$\O_Λ$ to be
\begin{equation}
  \label{eq:lattice-order}
  \O_Λ = \{ α ∈ K \;\mid\; αΛ ⊂ Λ \}.
\end{equation}
It is clear that $\O_Λ$ is a ring, and it is easy to show that it is
an order of $K$, and thus that $Λ$ is a fractional $\O_Λ$-ideal. %
Using Theorem~\ref{th:weierstrass-p} we associate to $Λ$ a complex
elliptic curve $E_Λ$; but then, by definition, $\O_Λ≃\End(E_Λ)$. %
Said otherwise, $E_Λ$ \emph{complex multiplication} by $\O_Λ$.

We have thus found a way to construct elliptic curves over the complex
numbers with complex multiplication by a specified order. %
Conversely, every curve with complex multiplication arises this way. %
To show this, we look at the set of all isomorphism classes of
elliptic curves with complex multiplication by a specified order $\O$,
which we will denote by $\Ell(\O)$. %
Because homothetic lattices give rise to isomorphic curves, fractional
ideals $\a$ and $c\a$ will be associated to isomorphic curves $E_\a$
and $E_{c\a}$ as long as $c≠0$. %
This justifies looking at fractional ideals modulo principal ideals.

\begin{definition}[Ideal class group]
  Let $\O$ be an order of a number field $K$. %
  Let $\mathcal{I}(\O)$ be the group of invertible fractional
  $\O$-ideals, and let $\mathcal{P}(\O)$ be the group of principal
  ideals. %

  The \emph{ideal class group} of $\O$ is the quotient group
  \[\Cl(\O) = \mathcal{I}(\O)/\mathcal{P}(\O).\]
  It is a finite Abelian group; its order is called the \emph{class
    number} of $\O$, and denoted by $h(\O)$.
\end{definition}

When $\O$ is the maximal order, $\Cl(\O)$ is also called the class
group of $K$. %
The class group is a fundamental object in \emph{class field theory}:
when $\O$ is the maximal order, it is isomorphic to the Galois group
of the maximal unramified Abelian extension of $K$, also called the
\emph{Hilbert class field} of $K$; more generally, non-maximal orders
are connected to ramified Abelian extensions of $K$. %
The next theorem highlights a fundamental connection between the class
group and the modular $j$-invariant, and thus to elliptic curves with
complex multiplication by $\O$.

\begin{theorem}
  Let $\O$ be an order of a number field $K$, and let
  $\a_1,\dots,\a_{h(\O)}$ be representatives of $\Cl(\O)$. %
  Then:
  \begin{itemize}
  \item $K(j(\a_i))$ is an Abelian extension of $K$;
  \item The $j(\a_i)$ are all conjugate over $K$;
  \item The Galois group of $K(j(\a_i))$ is isomorphic to $\Cl(\O)$;
  \item $[ℚ(j(\a_i)):ℚ] = [K(j(\a_i)):K] = h(\O)$;
  \item The $j(\a_i)$ are integral, their minimal polynomial is called
    the \emph{Hilbert class polynomial} of $\O$;
  \item $\Cl(\O)$ acts freely and transitively on $\Ell(\O)$, in
    particular $\#\Ell(\O) = h(\O)$.
  \end{itemize}
\end{theorem}
\begin{proof}
  See~\cite[Ch.~II]{silverman:advanced} and~\cite[Ch.~10]{lang1987elliptic}.
\end{proof}

Hence, we have completely characterized all elliptic curves with
complex multiplication by an order $\O$, up to isomorphism; in
particular, we now know that $j$-invariants with complex
multiplication (sometimes called \emph{singular $j$-invariants}) are
algebraic integers. %
In the next part, we shall say more on how $\Cl(\O)$ acts on the set
$\Ell(\O)$.

\begin{example}
  Let $\O=ℤ[i]$, so that $\O$ is the ring of integers of $ℚ(i)$. %
  It was already proven by Gauss that $ℤ[i]$ is a principal ideal
  domain, and thus that its class group is trivial. %
  Up to homothety, there is a unique lattice with order $ℤ[i]$, and
  one such representative is $ℤ[i]$ itself.

  Recall the definition of the Eisenstein series
  \[G_{2k}(Λ) = \sum_{ω∈Λ\setminus\{0\}} ω^{-2k}.\]
  But in our case $Λ=ℤ[i]$, thus $iΛ=Λ$, hence
  \[G_{2k}(Λ) = G_{2k}(iΛ) = i^{-2k}G_{2k}(Λ) = (-1)^kG_{2k}(Λ).\] In
  particular $G_6(Λ) = - G_6(Λ) = 0$, hence, by the definition of the
  modular $j$-invariant, $j(ℤ[i]) = 1728$.

  This shows that that the Hilbert class polynomial of $ℤ[i]$ is
  $X-1728$, and that the curve $E\;:\;y^2=x^3+x$ is the only curve
  over $ℂ$, up to isomorphism, with complex multiplication by
  $ℤ[i]$. %
  In particular, $ℤ[i]$ contains a subgroup of units $\{±1,±i\}$,
  which correspond to the four automorphisms generated by the map
  \begin{align*}
    ι : E &→ E,\\
    (x,y) &↦ (-x,iy).
  \end{align*}
\end{example}


\subsection{Complex multiplication for finite fields}
At this point, we have a complete characterization of complex
multiplication elliptic curves in characteristic $0$. %
What happens, then, in positive characteristic $p$? %

There are at least two ways in which we could construct elliptic
curves over a finite field with endomorphism ring larger than $ℤ$. %
One is to start from a complex multiplication elliptic curve $E$
defined over a number field $L$, and then reduce at a place\footnote{A
  \emph{place} is just a fancy name for a prime ideal of $L$.}
$\frak{p}$ over $p$. %
We write $\bar{E} = E(\frak{p})$ for the reduction of $E$ at the place
$\frak{p}$; if we do this carefully (for example, we must avoid
singular reductions), non-trivial endomorphisms of $E$ will descend to
non-trivial endomorphisms of $\bar{E}$. %

\begin{theorem}[Deuring]
  Let $E$ be an elliptic curve over a number field $L$, with complex
  multiplication by an order $\O⊂K$. %
  Let $\frak{p}$ be a place of $L$ over $p$, and assume that $E$
  has non-singular reduction $\bar{E}$ modulo $\frak{p}$. %
  The curve $\bar{E}$ is supersingular if and only if $p$ has only one
  prime of $K$ above it ($p$ ramifies or remains prime in $k$).

  Suppose that $p$ splits completely in $K$. %
  Let $f$ be the conductor of $\O$, and write $f = p^rf_0$, where
  $p\nmid f_0$. %
  Then:
  \begin{itemize}
  \item $\bar{E}$ has complex multiplication by the order in $K$ with
    conductor $f_0$.
  \item If $p\nmid f$, then the map $ω↦\omega(\frak{p})$ defines an
    isomorphism of $\End(E)$ and $\End(\bar{E})$.
  \end{itemize}
\end{theorem}
\begin{proof}
  See~\cite[Ch.~13]{lang1987elliptic}.
\end{proof}

Note that $p>2$ splits in $K$ if and only if the fundamental
discriminant $Δ_K$ of $K$ is is a square modulo $p$. %
To include the case $p=2$, we may use the Kronecker symbol
$\left(\frac{Δ_K}{p}\right)$, which is equal to $1$ if and only if $p$
splits. %

\begin{example}
  We have seen that the elliptic curve $E/ℚ$ defined by $y^2=x^3+x$
  has complex multiplication by $ℤ[i]$. %
  Assume $p>2$; by virtue of the theorem above, $E(p)$ is
  supersingular if and only if $(-4/p)=-1$, i.e., if and only if
  $p≡3 \bmod 4$.

  In particular, this implies that $-1$ is not a square modulo $p$,
  and thus that the automorphism $(x,y)↦(-x,iy)$ does not descend to
  an $\F_p$-automorphism of $E(p)$. %
  It does, however, descend to an $\F_{p^2}$-automorphism, showing
  that $\End(E(p))$ is not commutative.
\end{example}

Another approach is to directly construct a curve $E/\F_q$ so that its
Frobenius endomorphism is in the desired order. %
Recall that the Frobenius endomorphism $π$ satisfies a quadratic
equation
\[π^2 - tπ + q = 0,\] %
with discriminant $Δ_π=t^2-4q≤0$. %
Setting the case $Δ_π=0$ aside, $\End(E)$ necessarily contains a
subring $ℤ[π]$, isomorphic to an order of $ℚ(\sqrt{Δ_π})$. %
It turns out that these approach is essentially equivalent to the
previous one, as a famous theorem shows.

\begin{theorem}[Deuring's lifting theorem]
  Let $E_0$ be an elliptic curve in characteristic $p$, with an
  endomorphism $ω_o$ which is not trivial. %
  Then there exists an elliptic curve $E$ defined over a number field
  $L$, an endomorphism $ω$ of $E$, and a non-singular reduction of $E$
  at a place $\frak{p}$ of $L$ lying above $p$, such that $E_0$ is
  isomorphic to $E(\frak{p})$, and $ω_0$ corresponds to $ω(\frak{p})$
  under the isomorphism.
\end{theorem}
\begin{proof}
  See~\cite[Ch.~13]{lang1987elliptic}.
\end{proof}


\section*{Exercises}

\begin{exercise}
  Prove Proposition~\ref{th:j}.
\end{exercise}

\begin{exercise}
  Determine all the possible automorphisms of elliptic curves.
\end{exercise}

\begin{exercise}
  Prove Proposition~\ref{th:frob}.
\end{exercise}

\begin{exercise}
  Using Proposition~\ref{th:weil}, devise an algorithm to effectively
  compute $\#E(\F_{q^n})$ given $\#E(\F_q)$.
\end{exercise}

\begin{exercise}
  Prove Corollary~\ref{coro:isog-basic}
\end{exercise}

\begin{exercise}
  Let $K$ be a complex imaginary number field, $Λ⊂K$ a complex
  lattice, and $\O_Λ$ its order as defined in
  Eq.~\eqref{eq:lattice-order}. %
  Prove that $\O_Λ$ is an order of $K$.
\end{exercise}

\begin{exercise}
  Let $ω∈ℂ$ be a cube root of unity, the ring $ℤ[ω]$ is also known as
  the \emph{Eisenstein integers}. %
  Determine all elliptic curves with complex multiplication by $ℤ[ω]$.
\end{exercise}

\begin{exercise}
  Prove that $-163$ is not a square modulo all odd primes
  $<41$. (Hint: $ℚ(\sqrt{-163})$ has class number $1$).
\end{exercise}


%%%%%%%%%%%%%%%%%%%%%%%%%%%%%%%%%%%%%%%%%%%%%%%%%%%%%%

\clearpage
\part{Isogeny graphs}

%%%%%%%%%%%%%%%%%%%%%%%%%%%%%%%%%%%%%%%%%%%%%%%%%%%%%%

\clearpage
\part{Key exchange}


%%%%%%%%%%%%%%%%%%%%%%%%%%%%%%%%%%%%%%%%%%%%%%%%%%%%%%

\clearpage
\part{Signatures, other protocols, open problems}

%%%%%%%%%%%%%%%%%%%%%%%%%%%%%%%%%%%%%%%%%%%%%%%%%%%%%%
\clearpage
\bibliographystyle{plain}
\bibliography{wurzburg,isogenies_bib/isogenies}

\end{document}

% Local Variables:
% ispell-local-dictionary:"american"
% End:

%  LocalWords:  Isogeny projective affine Abelian invertible isogeny
%  LocalWords:  isomorphism endomorphism endomorphisms isogenies
%  LocalWords:  cardinality automorphisms cryptographic cryptosystem
%  LocalWords:  parallelize homothety Homothetic invariants Expander
%  LocalWords:  supersingular irreducibility cryptographically torsor
%  LocalWords:  expander expanders Schreier coprime quaternion monic
% LocalWords:  morphisms bijection subring morphism surjective
% LocalWords:  discriminants homothetic
